
% Default to the notebook output style




% Inherit from the specified cell style.





\documentclass{article}



    \usepackage{graphicx} % Used to insert images
    \usepackage{adjustbox} % Used to constrain images to a maximum size
    \usepackage{color} % Allow colors to be defined
    \usepackage{enumerate} % Needed for markdown enumerations to work
    \usepackage{geometry} % Used to adjust the document margins
    \usepackage{amsmath} % Equations
    \usepackage{amssymb} % Equations
    \usepackage{eurosym} % defines \euro
    \usepackage[mathletters]{ucs} % Extended unicode (utf-8) support
    \usepackage[utf8x]{inputenc} % Allow utf-8 characters in the tex document
    \usepackage{fancyvrb} % verbatim replacement that allows latex
    \usepackage{fancyhdr} % for header
    \usepackage{grffile} % extends the file name processing of package graphics
                         % to support a larger range
    % The hyperref package gives us a pdf with properly built
    % internal navigation ('pdf bookmarks' for the table of contents,
    % internal cross-reference links, web links for URLs, etc.)
    \usepackage{hyperref}
    \usepackage{longtable} % longtable support required by pandoc >1.10
    \usepackage{booktabs}  % table support for pandoc > 1.12.2




    \definecolor{orange}{cmyk}{0,0.4,0.8,0.2}
    \definecolor{darkorange}{rgb}{.71,0.21,0.01}
    \definecolor{darkgreen}{rgb}{.12,.54,.11}
    \definecolor{myteal}{rgb}{.26, .44, .56}
    \definecolor{gray}{gray}{0.45}
    \definecolor{lightgray}{gray}{.95}
    \definecolor{mediumgray}{gray}{.8}
    \definecolor{inputbackground}{rgb}{.95, .95, .85}
    \definecolor{outputbackground}{rgb}{.95, .95, .95}
    \definecolor{traceback}{rgb}{1, .95, .95}
    % ansi colors
    \definecolor{red}{rgb}{.6,0,0}
    \definecolor{green}{rgb}{0,.65,0}
    \definecolor{brown}{rgb}{0.6,0.6,0}
    \definecolor{blue}{rgb}{0,.145,.698}
    \definecolor{purple}{rgb}{.698,.145,.698}
    \definecolor{cyan}{rgb}{0,.698,.698}
    \definecolor{lightgray}{gray}{0.5}

    % bright ansi colors
    \definecolor{darkgray}{gray}{0.25}
    \definecolor{lightred}{rgb}{1.0,0.39,0.28}
    \definecolor{lightgreen}{rgb}{0.48,0.99,0.0}
    \definecolor{lightblue}{rgb}{0.53,0.81,0.92}
    \definecolor{lightpurple}{rgb}{0.87,0.63,0.87}
    \definecolor{lightcyan}{rgb}{0.5,1.0,0.83}

    % commands and environments needed by pandoc snippets
    % extracted from the output of `pandoc -s`
    \providecommand{\tightlist}{%
      \setlength{\itemsep}{0pt}\setlength{\parskip}{0pt}}
    \DefineVerbatimEnvironment{Highlighting}{Verbatim}{commandchars=\\\{\}}
    % Add ',fontsize=\small' for more characters per line
    \newenvironment{Shaded}{}{}
    \newcommand{\KeywordTok}[1]{\textcolor[rgb]{0.00,0.44,0.13}{\textbf{{#1}}}}
    \newcommand{\DataTypeTok}[1]{\textcolor[rgb]{0.56,0.13,0.00}{{#1}}}
    \newcommand{\DecValTok}[1]{\textcolor[rgb]{0.25,0.63,0.44}{{#1}}}
    \newcommand{\BaseNTok}[1]{\textcolor[rgb]{0.25,0.63,0.44}{{#1}}}
    \newcommand{\FloatTok}[1]{\textcolor[rgb]{0.25,0.63,0.44}{{#1}}}
    \newcommand{\CharTok}[1]{\textcolor[rgb]{0.25,0.44,0.63}{{#1}}}
    \newcommand{\StringTok}[1]{\textcolor[rgb]{0.25,0.44,0.63}{{#1}}}
    \newcommand{\CommentTok}[1]{\textcolor[rgb]{0.38,0.63,0.69}{\textit{{#1}}}}
    \newcommand{\OtherTok}[1]{\textcolor[rgb]{0.00,0.44,0.13}{{#1}}}
    \newcommand{\AlertTok}[1]{\textcolor[rgb]{1.00,0.00,0.00}{\textbf{{#1}}}}
    \newcommand{\FunctionTok}[1]{\textcolor[rgb]{0.02,0.16,0.49}{{#1}}}
    \newcommand{\RegionMarkerTok}[1]{{#1}}
    \newcommand{\ErrorTok}[1]{\textcolor[rgb]{1.00,0.00,0.00}{\textbf{{#1}}}}
    \newcommand{\NormalTok}[1]{{#1}}

    % Define a nice break command that doesn't care if a line doesn't already
    % exist.
    \def\br{\hspace*{\fill} \\* }
    % Math Jax compatability definitions
    \def\gt{>}
    \def\lt{<}

    % define header
    \pagestyle{fancy}

    \lhead{WLU CP468, December 11 2015}
    \chead{SGA -- Final Project}
    \rhead{George Lifchits (100691350)}


    % Document parameters
    \title{SGA}




    % Pygments definitions

\makeatletter
\def\PY@reset{\let\PY@it=\relax \let\PY@bf=\relax%
    \let\PY@ul=\relax \let\PY@tc=\relax%
    \let\PY@bc=\relax \let\PY@ff=\relax}
\def\PY@tok#1{\csname PY@tok@#1\endcsname}
\def\PY@toks#1+{\ifx\relax#1\empty\else%
    \PY@tok{#1}\expandafter\PY@toks\fi}
\def\PY@do#1{\PY@bc{\PY@tc{\PY@ul{%
    \PY@it{\PY@bf{\PY@ff{#1}}}}}}}
\def\PY#1#2{\PY@reset\PY@toks#1+\relax+\PY@do{#2}}

\expandafter\def\csname PY@tok@ge\endcsname{\let\PY@it=\textit}
\expandafter\def\csname PY@tok@c1\endcsname{\let\PY@it=\textit\def\PY@tc##1{\textcolor[rgb]{0.25,0.50,0.50}{##1}}}
\expandafter\def\csname PY@tok@nn\endcsname{\let\PY@bf=\textbf\def\PY@tc##1{\textcolor[rgb]{0.00,0.00,1.00}{##1}}}
\expandafter\def\csname PY@tok@s1\endcsname{\def\PY@tc##1{\textcolor[rgb]{0.73,0.13,0.13}{##1}}}
\expandafter\def\csname PY@tok@ne\endcsname{\let\PY@bf=\textbf\def\PY@tc##1{\textcolor[rgb]{0.82,0.25,0.23}{##1}}}
\expandafter\def\csname PY@tok@se\endcsname{\let\PY@bf=\textbf\def\PY@tc##1{\textcolor[rgb]{0.73,0.40,0.13}{##1}}}
\expandafter\def\csname PY@tok@mb\endcsname{\def\PY@tc##1{\textcolor[rgb]{0.40,0.40,0.40}{##1}}}
\expandafter\def\csname PY@tok@nd\endcsname{\def\PY@tc##1{\textcolor[rgb]{0.67,0.13,1.00}{##1}}}
\expandafter\def\csname PY@tok@na\endcsname{\def\PY@tc##1{\textcolor[rgb]{0.49,0.56,0.16}{##1}}}
\expandafter\def\csname PY@tok@sc\endcsname{\def\PY@tc##1{\textcolor[rgb]{0.73,0.13,0.13}{##1}}}
\expandafter\def\csname PY@tok@s2\endcsname{\def\PY@tc##1{\textcolor[rgb]{0.73,0.13,0.13}{##1}}}
\expandafter\def\csname PY@tok@err\endcsname{\def\PY@bc##1{\setlength{\fboxsep}{0pt}\fcolorbox[rgb]{1.00,0.00,0.00}{1,1,1}{\strut ##1}}}
\expandafter\def\csname PY@tok@kt\endcsname{\def\PY@tc##1{\textcolor[rgb]{0.69,0.00,0.25}{##1}}}
\expandafter\def\csname PY@tok@ss\endcsname{\def\PY@tc##1{\textcolor[rgb]{0.10,0.09,0.49}{##1}}}
\expandafter\def\csname PY@tok@il\endcsname{\def\PY@tc##1{\textcolor[rgb]{0.40,0.40,0.40}{##1}}}
\expandafter\def\csname PY@tok@sb\endcsname{\def\PY@tc##1{\textcolor[rgb]{0.73,0.13,0.13}{##1}}}
\expandafter\def\csname PY@tok@nv\endcsname{\def\PY@tc##1{\textcolor[rgb]{0.10,0.09,0.49}{##1}}}
\expandafter\def\csname PY@tok@m\endcsname{\def\PY@tc##1{\textcolor[rgb]{0.40,0.40,0.40}{##1}}}
\expandafter\def\csname PY@tok@mi\endcsname{\def\PY@tc##1{\textcolor[rgb]{0.40,0.40,0.40}{##1}}}
\expandafter\def\csname PY@tok@nl\endcsname{\def\PY@tc##1{\textcolor[rgb]{0.63,0.63,0.00}{##1}}}
\expandafter\def\csname PY@tok@sh\endcsname{\def\PY@tc##1{\textcolor[rgb]{0.73,0.13,0.13}{##1}}}
\expandafter\def\csname PY@tok@kp\endcsname{\def\PY@tc##1{\textcolor[rgb]{0.00,0.50,0.00}{##1}}}
\expandafter\def\csname PY@tok@nt\endcsname{\let\PY@bf=\textbf\def\PY@tc##1{\textcolor[rgb]{0.00,0.50,0.00}{##1}}}
\expandafter\def\csname PY@tok@o\endcsname{\def\PY@tc##1{\textcolor[rgb]{0.40,0.40,0.40}{##1}}}
\expandafter\def\csname PY@tok@sd\endcsname{\let\PY@it=\textit\def\PY@tc##1{\textcolor[rgb]{0.73,0.13,0.13}{##1}}}
\expandafter\def\csname PY@tok@vi\endcsname{\def\PY@tc##1{\textcolor[rgb]{0.10,0.09,0.49}{##1}}}
\expandafter\def\csname PY@tok@gd\endcsname{\def\PY@tc##1{\textcolor[rgb]{0.63,0.00,0.00}{##1}}}
\expandafter\def\csname PY@tok@sx\endcsname{\def\PY@tc##1{\textcolor[rgb]{0.00,0.50,0.00}{##1}}}
\expandafter\def\csname PY@tok@sr\endcsname{\def\PY@tc##1{\textcolor[rgb]{0.73,0.40,0.53}{##1}}}
\expandafter\def\csname PY@tok@mf\endcsname{\def\PY@tc##1{\textcolor[rgb]{0.40,0.40,0.40}{##1}}}
\expandafter\def\csname PY@tok@kn\endcsname{\let\PY@bf=\textbf\def\PY@tc##1{\textcolor[rgb]{0.00,0.50,0.00}{##1}}}
\expandafter\def\csname PY@tok@s\endcsname{\def\PY@tc##1{\textcolor[rgb]{0.73,0.13,0.13}{##1}}}
\expandafter\def\csname PY@tok@cp\endcsname{\def\PY@tc##1{\textcolor[rgb]{0.74,0.48,0.00}{##1}}}
\expandafter\def\csname PY@tok@nc\endcsname{\let\PY@bf=\textbf\def\PY@tc##1{\textcolor[rgb]{0.00,0.00,1.00}{##1}}}
\expandafter\def\csname PY@tok@vc\endcsname{\def\PY@tc##1{\textcolor[rgb]{0.10,0.09,0.49}{##1}}}
\expandafter\def\csname PY@tok@gp\endcsname{\let\PY@bf=\textbf\def\PY@tc##1{\textcolor[rgb]{0.00,0.00,0.50}{##1}}}
\expandafter\def\csname PY@tok@kd\endcsname{\let\PY@bf=\textbf\def\PY@tc##1{\textcolor[rgb]{0.00,0.50,0.00}{##1}}}
\expandafter\def\csname PY@tok@k\endcsname{\let\PY@bf=\textbf\def\PY@tc##1{\textcolor[rgb]{0.00,0.50,0.00}{##1}}}
\expandafter\def\csname PY@tok@no\endcsname{\def\PY@tc##1{\textcolor[rgb]{0.53,0.00,0.00}{##1}}}
\expandafter\def\csname PY@tok@bp\endcsname{\def\PY@tc##1{\textcolor[rgb]{0.00,0.50,0.00}{##1}}}
\expandafter\def\csname PY@tok@gh\endcsname{\let\PY@bf=\textbf\def\PY@tc##1{\textcolor[rgb]{0.00,0.00,0.50}{##1}}}
\expandafter\def\csname PY@tok@nf\endcsname{\def\PY@tc##1{\textcolor[rgb]{0.00,0.00,1.00}{##1}}}
\expandafter\def\csname PY@tok@gu\endcsname{\let\PY@bf=\textbf\def\PY@tc##1{\textcolor[rgb]{0.50,0.00,0.50}{##1}}}
\expandafter\def\csname PY@tok@gt\endcsname{\def\PY@tc##1{\textcolor[rgb]{0.00,0.27,0.87}{##1}}}
\expandafter\def\csname PY@tok@mo\endcsname{\def\PY@tc##1{\textcolor[rgb]{0.40,0.40,0.40}{##1}}}
\expandafter\def\csname PY@tok@w\endcsname{\def\PY@tc##1{\textcolor[rgb]{0.73,0.73,0.73}{##1}}}
\expandafter\def\csname PY@tok@gi\endcsname{\def\PY@tc##1{\textcolor[rgb]{0.00,0.63,0.00}{##1}}}
\expandafter\def\csname PY@tok@kc\endcsname{\let\PY@bf=\textbf\def\PY@tc##1{\textcolor[rgb]{0.00,0.50,0.00}{##1}}}
\expandafter\def\csname PY@tok@gs\endcsname{\let\PY@bf=\textbf}
\expandafter\def\csname PY@tok@vg\endcsname{\def\PY@tc##1{\textcolor[rgb]{0.10,0.09,0.49}{##1}}}
\expandafter\def\csname PY@tok@si\endcsname{\let\PY@bf=\textbf\def\PY@tc##1{\textcolor[rgb]{0.73,0.40,0.53}{##1}}}
\expandafter\def\csname PY@tok@gr\endcsname{\def\PY@tc##1{\textcolor[rgb]{1.00,0.00,0.00}{##1}}}
\expandafter\def\csname PY@tok@cm\endcsname{\let\PY@it=\textit\def\PY@tc##1{\textcolor[rgb]{0.25,0.50,0.50}{##1}}}
\expandafter\def\csname PY@tok@cs\endcsname{\let\PY@it=\textit\def\PY@tc##1{\textcolor[rgb]{0.25,0.50,0.50}{##1}}}
\expandafter\def\csname PY@tok@ow\endcsname{\let\PY@bf=\textbf\def\PY@tc##1{\textcolor[rgb]{0.67,0.13,1.00}{##1}}}
\expandafter\def\csname PY@tok@mh\endcsname{\def\PY@tc##1{\textcolor[rgb]{0.40,0.40,0.40}{##1}}}
\expandafter\def\csname PY@tok@ni\endcsname{\let\PY@bf=\textbf\def\PY@tc##1{\textcolor[rgb]{0.60,0.60,0.60}{##1}}}
\expandafter\def\csname PY@tok@nb\endcsname{\def\PY@tc##1{\textcolor[rgb]{0.00,0.50,0.00}{##1}}}
\expandafter\def\csname PY@tok@kr\endcsname{\let\PY@bf=\textbf\def\PY@tc##1{\textcolor[rgb]{0.00,0.50,0.00}{##1}}}
\expandafter\def\csname PY@tok@c\endcsname{\let\PY@it=\textit\def\PY@tc##1{\textcolor[rgb]{0.25,0.50,0.50}{##1}}}
\expandafter\def\csname PY@tok@go\endcsname{\def\PY@tc##1{\textcolor[rgb]{0.53,0.53,0.53}{##1}}}

\def\PYZbs{\char`\\}
\def\PYZus{\char`\_}
\def\PYZob{\char`\{}
\def\PYZcb{\char`\}}
\def\PYZca{\char`\^}
\def\PYZam{\char`\&}
\def\PYZlt{\char`\<}
\def\PYZgt{\char`\>}
\def\PYZsh{\char`\#}
\def\PYZpc{\char`\%}
\def\PYZdl{\char`\$}
\def\PYZhy{\char`\-}
\def\PYZsq{\char`\'}
\def\PYZdq{\char`\"}
\def\PYZti{\char`\~}
% for compatibility with earlier versions
\def\PYZat{@}
\def\PYZlb{[}
\def\PYZrb{]}
\makeatother


    % Exact colors from NB
    \definecolor{incolor}{rgb}{0.0, 0.0, 0.5}
    \definecolor{outcolor}{rgb}{0.545, 0.0, 0.0}




    % Prevent overflowing lines due to hard-to-break entities
    \sloppy
    % Setup hyperref package
    \hypersetup{
      breaklinks=true,  % so long urls are correctly broken across lines
      colorlinks=true,
      urlcolor=blue,
      linkcolor=darkorange,
      citecolor=darkgreen,
      }
    % Slightly bigger margins than the latex defaults

    \geometry{verbose,tmargin=1in,bmargin=1in,lmargin=1in,rmargin=1in}



    \begin{document}

    \title{Simple Genetic Algorithm}
    \author{George Lifchits (100691350) \\
    CP468: Artificial Intelligence}


    \maketitle




    \section{Running this code}\label{running-this-code}

This project is implemented in a Jupyter (a.k.a. IPython) notebook, with
code written in Python 3.

Jupyter allows the interspersal of simple formatted text and executable
blocks of code with visible output. It also offers good integration with
matplotlib, a Python plotting library, such that rendered graphs and
plots are output directly in the notebook.

More information on Jupyter can be found on their website:
\url{https://jupyter.org/}, and instructions on installing it can be
found here:
\url{http://jupyter.readthedocs.org/en/latest/install.html\#new-to-python-and-jupyter}.

Once the Jupyter environment is installed and configured with a Python 3
kernel, the only requirement to run this notebook interactively is to
start a Jupyter server instance in the same folder as the
\texttt{SGA.ipynb} file (on Mac, I do this by opening a bash shell and
typing \texttt{jupyter\ notebook}), and open the notebook in the browser
window that opens.

    \section{SGA Implementation}\label{sga-implementation}

We start off with some imports for plotting graphs.

    \begin{Verbatim}[commandchars=\\\{\}]
{\color{incolor}In [{\color{incolor}1}]:} \PY{k+kn}{import} \PY{n+nn}{matplotlib}
        \PY{k+kn}{from} \PY{n+nn}{matplotlib} \PY{k}{import} \PY{n}{cm}
        \PY{k+kn}{import} \PY{n+nn}{matplotlib}\PY{n+nn}{.}\PY{n+nn}{pyplot} \PY{k}{as} \PY{n+nn}{plt}
        \PY{k+kn}{from} \PY{n+nn}{mpl\PYZus{}toolkits}\PY{n+nn}{.}\PY{n+nn}{mplot3d} \PY{k}{import} \PY{n}{Axes3D}
        \PY{o}{\PYZpc{}}\PY{k}{matplotlib} inline
        \PY{o}{\PYZpc{}}\PY{k}{config} InlineBackend.figure\PYZus{}formats = [\PYZsq{}png\PYZsq{}]
        \PY{n}{matplotlib}\PY{o}{.}\PY{n}{rcParams}\PY{p}{[}\PY{l+s}{\PYZsq{}}\PY{l+s}{figure.figsize}\PY{l+s}{\PYZsq{}}\PY{p}{]} \PY{o}{=} \PY{p}{(}\PY{l+m+mf}{18.0}\PY{p}{,} \PY{l+m+mf}{8.0}\PY{p}{)}
\end{Verbatim}

    SGA has two important entities which have to be modelled in a code implementation: the
    \emph{chromosome} and the \emph{population}.  Here I implemented the chromosome as a
    Python string, where each character is either 0 or 1. Populations are a Python list of
    strings.  The following are helpers, including methods used to initialize the
    population.

    \begin{Verbatim}[commandchars=\\\{\}]
{\color{incolor}In [{\color{incolor}2}]:} \PY{k+kn}{import} \PY{n+nn}{random}


        \PY{k}{def} \PY{n+nf}{coin}\PY{p}{(}\PY{n}{prob}\PY{p}{)}\PY{p}{:}
            \PY{l+s+sd}{\PYZdq{}\PYZdq{}\PYZdq{}}
        \PY{l+s+sd}{    Performs a biased coin toss.}
        \PY{l+s+sd}{    :param prob: [0 ≤ float ≤ 1]}
        \PY{l+s+sd}{    :returns: [bool] True with probability `prob` and otherwise False}
        \PY{l+s+sd}{    \PYZdq{}\PYZdq{}\PYZdq{}}
            \PY{c}{\PYZsh{} random.random() yields a float between 0 and 1}
            \PY{k}{return} \PY{n}{random}\PY{o}{.}\PY{n}{random}\PY{p}{(}\PY{p}{)} \PY{o}{\PYZlt{}} \PY{n}{prob}


        \PY{k}{def} \PY{n+nf}{random\PYZus{}string}\PY{p}{(}\PY{n}{length}\PY{p}{)}\PY{p}{:}
            \PY{l+s+sd}{\PYZdq{}\PYZdq{}\PYZdq{}}
        \PY{l+s+sd}{    :param length: [int] length of random string}
        \PY{l+s+sd}{    :returns: [string] random string consisting of \PYZdq{}0\PYZdq{} and \PYZdq{}1\PYZdq{}}
        \PY{l+s+sd}{    \PYZdq{}\PYZdq{}\PYZdq{}}
            \PY{k}{return} \PY{l+s}{\PYZsq{}}\PY{l+s}{\PYZsq{}}\PY{o}{.}\PY{n}{join}\PY{p}{(}\PY{l+s}{\PYZsq{}}\PY{l+s}{0}\PY{l+s}{\PYZsq{}} \PY{k}{if} \PY{n}{coin}\PY{p}{(}\PY{l+m+mf}{0.5}\PY{p}{)} \PY{k}{else} \PY{l+s}{\PYZsq{}}\PY{l+s}{1}\PY{l+s}{\PYZsq{}} \PY{k}{for} \PY{n}{\PYZus{}} \PY{o+ow}{in} \PY{n+nb}{range}\PY{p}{(}\PY{n}{length}\PY{p}{)}\PY{p}{)}


        \PY{k}{def} \PY{n+nf}{generate\PYZus{}random\PYZus{}population}\PY{p}{(}\PY{n}{number}\PY{p}{,} \PY{n}{length}\PY{p}{)}\PY{p}{:}
            \PY{l+s+sd}{\PYZdq{}\PYZdq{}\PYZdq{}}
        \PY{l+s+sd}{    This function is used to generate the first population.}
        \PY{l+s+sd}{    This implementation ensure that chromosomes in the initial population}
        \PY{l+s+sd}{    are uniformly pseudo\PYZhy{}random!}
        \PY{l+s+sd}{    }
        \PY{l+s+sd}{    :param number: [int] number of strings to return}
        \PY{l+s+sd}{    :param length: [int] length of the strings to return}
        \PY{l+s+sd}{    :returns: List[str] list of random binary strings}
        \PY{l+s+sd}{    \PYZdq{}\PYZdq{}\PYZdq{}}
            \PY{k}{return} \PY{p}{[}\PY{n}{random\PYZus{}string}\PY{p}{(}\PY{n}{length}\PY{p}{)} \PY{k}{for} \PY{n}{\PYZus{}} \PY{o+ow}{in} \PY{n+nb}{range}\PY{p}{(}\PY{n}{number}\PY{p}{)}\PY{p}{]}
\end{Verbatim}

    \subsection{Reproduction}\label{reproduction}

One of the main three operators in SGA. It takes a population of
chromosomes and generates a new population.

\begin{itemize}
\tightlist
\item
  Each chromosome in the population is assigned a weight, based off the
  chromosome's \textbf{fitness}
\item
  The weights define how likely the member is going to be picked for the
  next population
\item
  This is called \emph{biased roulette} selection
\end{itemize}

    \begin{Verbatim}[commandchars=\\\{\}]
{\color{incolor}In [{\color{incolor}3}]:} \PY{n}{MIN} \PY{o}{=} \PY{l+m+mi}{0}
        \PY{n}{MAX} \PY{o}{=} \PY{l+m+mi}{1}


        \PY{k}{def} \PY{n+nf}{reproduction}\PY{p}{(}\PY{n}{population}\PY{p}{,} \PY{n}{fitness\PYZus{}func}\PY{p}{,} \PY{n}{min\PYZus{}or\PYZus{}max}\PY{o}{=}\PY{n}{MAX}\PY{p}{)}\PY{p}{:}
            \PY{l+s+sd}{\PYZdq{}\PYZdq{}\PYZdq{}}
        \PY{l+s+sd}{    Produces a new population from biased roulette reproduction of the }
        \PY{l+s+sd}{    given population.}
        \PY{l+s+sd}{    :param population: [List[str]]}
        \PY{l+s+sd}{    :param fitness\PYZus{}func: [function: number \PYZgt{} 0]}
        \PY{l+s+sd}{    :param min\PYZus{}or\PYZus{}max: \PYZob{}MIN, MAX\PYZcb{}}
        \PY{l+s+sd}{    :returns: [List[str]]}
        \PY{l+s+sd}{    \PYZdq{}\PYZdq{}\PYZdq{}}
            \PY{c}{\PYZsh{} First, we define the probability density (roulette weights) for each}
            \PY{c}{\PYZsh{} member in our given population. }

            \PY{n}{min\PYZus{}fitness} \PY{o}{=} \PY{n+nb}{min}\PY{p}{(}\PY{n}{fitness\PYZus{}func}\PY{p}{(}\PY{n}{m}\PY{p}{)} \PY{k}{for} \PY{n}{m} \PY{o+ow}{in} \PY{n}{population}\PY{p}{)}

            \PY{k}{def} \PY{n+nf}{compute\PYZus{}weight}\PY{p}{(}\PY{n}{m}\PY{p}{)}\PY{p}{:}
                \PY{l+s+sd}{\PYZdq{}\PYZdq{}\PYZdq{}}
        \PY{l+s+sd}{        Subroutine which computes the weight of the biased roulette, which }
        \PY{l+s+sd}{        is agnostic of the fitness function. In particular, it will invert}
        \PY{l+s+sd}{        the fitness value if we are seeking a minimum. Member `m` has weight}
        \PY{l+s+sd}{        that is commensurate with its distance from the member with lowest}
        \PY{l+s+sd}{        fitness in the population.}
        \PY{l+s+sd}{        :param m: [str] member}
        \PY{l+s+sd}{        \PYZdq{}\PYZdq{}\PYZdq{}}
                \PY{n}{fitness} \PY{o}{=} \PY{n}{fitness\PYZus{}func}\PY{p}{(}\PY{n}{m}\PY{p}{)}

                \PY{k}{if} \PY{n}{min\PYZus{}or\PYZus{}max} \PY{o}{==} \PY{n}{MAX}\PY{p}{:}
                    \PY{k}{return} \PY{n}{fitness} \PY{o}{\PYZhy{}} \PY{n}{min\PYZus{}fitness} \PY{o}{+} \PY{l+m+mi}{1}

                \PY{k}{elif} \PY{n}{min\PYZus{}or\PYZus{}max} \PY{o}{==} \PY{n}{MIN}\PY{p}{:}
                    \PY{k}{return} \PY{l+m+mi}{1} \PY{o}{/} \PY{p}{(}\PY{n}{fitness} \PY{o}{\PYZhy{}} \PY{n}{min\PYZus{}fitness} \PY{o}{+} \PY{l+m+mi}{1}\PY{p}{)}

            \PY{c}{\PYZsh{} Here we normalize the weights to be proportions of the total weighting}
            \PY{n}{weights} \PY{o}{=} \PY{p}{[}\PY{p}{(}\PY{n}{m}\PY{p}{,} \PY{n}{compute\PYZus{}weight}\PY{p}{(}\PY{n}{m}\PY{p}{)}\PY{p}{)} \PY{k}{for} \PY{n}{m} \PY{o+ow}{in} \PY{n}{population}\PY{p}{]}
            \PY{n}{total\PYZus{}weights} \PY{o}{=} \PY{n+nb}{sum}\PY{p}{(}\PY{n}{w} \PY{k}{for} \PY{n}{m}\PY{p}{,} \PY{n}{w} \PY{o+ow}{in} \PY{n}{weights}\PY{p}{)}
            \PY{n}{pdf} \PY{o}{=} \PY{p}{[}\PY{p}{(}\PY{n}{m}\PY{p}{,} \PY{n}{w}\PY{o}{/}\PY{n}{total\PYZus{}weights}\PY{p}{)} \PY{k}{for} \PY{n}{m}\PY{p}{,} \PY{n}{w} \PY{o+ow}{in} \PY{n}{weights}\PY{p}{]}

            \PY{c}{\PYZsh{} Now we pick members for the new population.}
            \PY{c}{\PYZsh{} We pick the same number of members as the provided population.}
            \PY{n}{new\PYZus{}population} \PY{o}{=} \PY{p}{[}\PY{p}{]}
            \PY{k}{for} \PY{n}{i} \PY{o+ow}{in} \PY{n+nb}{range}\PY{p}{(}\PY{n+nb}{len}\PY{p}{(}\PY{n}{population}\PY{p}{)}\PY{p}{)}\PY{p}{:}
                \PY{n}{rand} \PY{o}{=} \PY{n}{random}\PY{o}{.}\PY{n}{random}\PY{p}{(}\PY{p}{)}
                \PY{n}{cumul} \PY{o}{=} \PY{l+m+mi}{0}
                \PY{k}{for} \PY{n}{member}\PY{p}{,} \PY{n}{end\PYZus{}interval} \PY{o+ow}{in} \PY{n}{pdf}\PY{p}{:}
                    \PY{n}{cumul} \PY{o}{+}\PY{o}{=} \PY{n}{end\PYZus{}interval}
                    \PY{k}{if} \PY{n}{rand} \PY{o}{\PYZlt{}}\PY{o}{=} \PY{n}{cumul}\PY{p}{:}
                        \PY{n}{new\PYZus{}population}\PY{o}{.}\PY{n}{append}\PY{p}{(}\PY{n}{member}\PY{p}{)}
                        \PY{k}{break} \PY{c}{\PYZsh{} generate next member}

            \PY{k}{return} \PY{n}{new\PYZus{}population}
\end{Verbatim}

    \subsection{Crossover}\label{crossover}

\begin{itemize}
\tightlist
\item
  Take pairs from the population
\item
  For each pair in the population, probability \(P_c\) is the chance
  that any pair will be crossed over
\end{itemize}

What is cross over?

\begin{itemize}
\tightlist
\item
  Pick a random index, split the strings into ``head'' and ``tail''
\item
  Take the head of the first and tail of the second, and vice versa
\end{itemize}

\textbf{Example:}

\begin{itemize}
\tightlist
\item
  \texttt{aaabbb} and \texttt{xxxyyy}
\end{itemize}

Crossover at randomly chosen index 2:

\begin{itemize}
\tightlist
\item
  \texttt{aaa\textbar{}yyy} and \texttt{xxx\textbar{}bbb}
\end{itemize}

    \begin{Verbatim}[commandchars=\\\{\}]
{\color{incolor}In [{\color{incolor}4}]:} \PY{k}{def} \PY{n+nf}{crossover}\PY{p}{(}\PY{n}{string1}\PY{p}{,} \PY{n}{string2}\PY{p}{,} \PY{n}{index}\PY{p}{)}\PY{p}{:}
            \PY{l+s+sd}{\PYZdq{}\PYZdq{}\PYZdq{} Performs crossover on two strings at given index \PYZdq{}\PYZdq{}\PYZdq{}}
            \PY{n}{head1}\PY{p}{,} \PY{n}{tail1} \PY{o}{=} \PY{n}{string1}\PY{p}{[}\PY{p}{:}\PY{n}{index}\PY{p}{]}\PY{p}{,} \PY{n}{string1}\PY{p}{[}\PY{n}{index}\PY{p}{:}\PY{p}{]}
            \PY{n}{head2}\PY{p}{,} \PY{n}{tail2} \PY{o}{=} \PY{n}{string2}\PY{p}{[}\PY{p}{:}\PY{n}{index}\PY{p}{]}\PY{p}{,} \PY{n}{string2}\PY{p}{[}\PY{n}{index}\PY{p}{:}\PY{p}{]}
            \PY{k}{return} \PY{n}{head1}\PY{o}{+}\PY{n}{tail2}\PY{p}{,} \PY{n}{head2}\PY{o}{+}\PY{n}{tail1}


        \PY{k}{def} \PY{n+nf}{population\PYZus{}crossover}\PY{p}{(}\PY{n}{population}\PY{p}{,} \PY{n}{crossover\PYZus{}probability}\PY{p}{)}\PY{p}{:}
            \PY{l+s+sd}{\PYZdq{}\PYZdq{}\PYZdq{}}
        \PY{l+s+sd}{    Performs crossover on an entire population.}
        \PY{l+s+sd}{    :param population: List[str]}
        \PY{l+s+sd}{    :param crossover\PYZus{}probability: [0 ≤ float ≤ 1] }
        \PY{l+s+sd}{        chance that any pair will be crossed over}
        \PY{l+s+sd}{    :returns: List[str] }
        \PY{l+s+sd}{        new population with possibly some members crossed over}
        \PY{l+s+sd}{    \PYZdq{}\PYZdq{}\PYZdq{}}
            \PY{n}{pairs} \PY{o}{=} \PY{p}{[}\PY{p}{]}
            \PY{n}{new\PYZus{}population} \PY{o}{=} \PY{p}{[}\PY{p}{]}
            \PY{k}{while} \PY{n+nb}{len}\PY{p}{(}\PY{n}{population}\PY{p}{)} \PY{o}{\PYZgt{}} \PY{l+m+mi}{1}\PY{p}{:}
                \PY{n}{pairs}\PY{o}{.}\PY{n}{append}\PY{p}{(}\PY{p}{(}\PY{n}{population}\PY{o}{.}\PY{n}{pop}\PY{p}{(}\PY{p}{)}\PY{p}{,} \PY{n}{population}\PY{o}{.}\PY{n}{pop}\PY{p}{(}\PY{p}{)}\PY{p}{)}\PY{p}{)}
            \PY{k}{if} \PY{n+nb}{len}\PY{p}{(}\PY{n}{population}\PY{p}{)} \PY{o}{==} \PY{l+m+mi}{1}\PY{p}{:}
                \PY{n}{new\PYZus{}population}\PY{o}{.}\PY{n}{append}\PY{p}{(}\PY{n}{population}\PY{o}{.}\PY{n}{pop}\PY{p}{(}\PY{p}{)}\PY{p}{)}

            \PY{k}{for} \PY{n}{s1}\PY{p}{,} \PY{n}{s2} \PY{o+ow}{in} \PY{n}{pairs}\PY{p}{:}
                \PY{k}{if} \PY{o+ow}{not} \PY{n}{coin}\PY{p}{(}\PY{n}{crossover\PYZus{}probability}\PY{p}{)}\PY{p}{:}
                    \PY{c}{\PYZsh{} don\PYZsq{}t perform crossover, just add the original pair}
                    \PY{n}{new\PYZus{}population} \PY{o}{+}\PY{o}{=} \PY{p}{[}\PY{n}{s1}\PY{p}{,} \PY{n}{s2}\PY{p}{]}
                    \PY{k}{continue}
                \PY{n}{idx} \PY{o}{=} \PY{n}{random}\PY{o}{.}\PY{n}{randint}\PY{p}{(}\PY{l+m+mi}{1}\PY{p}{,} \PY{n+nb}{len}\PY{p}{(}\PY{n}{s1}\PY{p}{)}\PY{o}{\PYZhy{}}\PY{l+m+mi}{1}\PY{p}{)} \PY{c}{\PYZsh{} select crossover index}
                \PY{n}{new\PYZus{}s1}\PY{p}{,} \PY{n}{new\PYZus{}s2} \PY{o}{=} \PY{n}{crossover}\PY{p}{(}\PY{n}{s1}\PY{p}{,} \PY{n}{s2}\PY{p}{,} \PY{n}{idx}\PY{p}{)}
                \PY{n}{new\PYZus{}population}\PY{o}{.}\PY{n}{append}\PY{p}{(}\PY{n}{new\PYZus{}s1}\PY{p}{)}
                \PY{n}{new\PYZus{}population}\PY{o}{.}\PY{n}{append}\PY{p}{(}\PY{n}{new\PYZus{}s2}\PY{p}{)}
            \PY{k}{return} \PY{n}{new\PYZus{}population}
\end{Verbatim}

    \subsection{Mutation}\label{mutation}

Creates perturbations in the population to find some chromosomes that
are not available from the crossover operator.

For every chromosome in the population, for every bit in the chromosome,
the bit will be flipped with a probability \(P_m\)

    \begin{Verbatim}[commandchars=\\\{\}]
{\color{incolor}In [{\color{incolor}5}]:} \PY{k}{def} \PY{n+nf}{mutation}\PY{p}{(}\PY{n}{string}\PY{p}{,} \PY{n}{probability}\PY{p}{)}\PY{p}{:}
            \PY{l+s+sd}{\PYZdq{}\PYZdq{}\PYZdq{}}
        \PY{l+s+sd}{    :param string: the binary string to mutate}
        \PY{l+s+sd}{    :param probability: [0 ≤ float ≤ 1] }
        \PY{l+s+sd}{        the probability of any character being flipped}
        \PY{l+s+sd}{    :returns: [str] }
        \PY{l+s+sd}{        just the input string, possibly with some bits flipped}
        \PY{l+s+sd}{    \PYZdq{}\PYZdq{}\PYZdq{}}
            \PY{n}{flipped} \PY{o}{=} \PY{k}{lambda} \PY{n}{x}\PY{p}{:} \PY{l+s}{\PYZsq{}}\PY{l+s}{1}\PY{l+s}{\PYZsq{}} \PY{k}{if} \PY{n}{x} \PY{o+ow}{is} \PY{l+s}{\PYZsq{}}\PY{l+s}{0}\PY{l+s}{\PYZsq{}} \PY{k}{else} \PY{l+s}{\PYZsq{}}\PY{l+s}{0}\PY{l+s}{\PYZsq{}}
            \PY{n}{chars} \PY{o}{=} \PY{p}{(}\PY{n}{flipped}\PY{p}{(}\PY{n}{char}\PY{p}{)} \PY{k}{if} \PY{n}{coin}\PY{p}{(}\PY{n}{probability}\PY{p}{)} \PY{k}{else} \PY{n}{char} \PY{k}{for} \PY{n}{char} \PY{o+ow}{in} \PY{n}{string}\PY{p}{)}
            \PY{k}{return} \PY{l+s}{\PYZsq{}}\PY{l+s}{\PYZsq{}}\PY{o}{.}\PY{n}{join}\PY{p}{(}\PY{n}{chars}\PY{p}{)}

        \PY{k}{def} \PY{n+nf}{mutate\PYZus{}population}\PY{p}{(}\PY{n}{population}\PY{p}{,} \PY{n}{prob}\PY{p}{)}\PY{p}{:}
            \PY{l+s+sd}{\PYZdq{}\PYZdq{}\PYZdq{}}
        \PY{l+s+sd}{    :param population: [List[str]] }
        \PY{l+s+sd}{        population of binary strings}
        \PY{l+s+sd}{    :returns: [List[str]] }
        \PY{l+s+sd}{        just the input population with some members possibly mutated}
        \PY{l+s+sd}{    \PYZdq{}\PYZdq{}\PYZdq{}}
            \PY{k}{return} \PY{p}{[}\PY{n}{mutation}\PY{p}{(}\PY{n}{m}\PY{p}{,} \PY{n}{prob}\PY{p}{)} \PY{k}{for} \PY{n}{m} \PY{o+ow}{in} \PY{n}{population}\PY{p}{]}
\end{Verbatim}

    \subsection{Main loop}\label{main-loop}

The main loop of SGA is very small. It runs for any number of
\emph{eras} (provided as parameter). In each era, we:

\begin{enumerate}
\def\labelenumi{\arabic{enumi}.}
\tightlist
\item
  perform \textbf{reproduction} to create a new population from the old
  population
\item
  perform \textbf{crossover} on the population
\item
  perform \textbf{mutation} on the population
\end{enumerate}

We save each population so that we can plot how the populations change
after each era.

    \begin{Verbatim}[commandchars=\\\{\}]
{\color{incolor}In [{\color{incolor}6}]:} \PY{k}{def} \PY{n+nf}{run\PYZus{}genetic\PYZus{}algorithm}\PY{p}{(}\PY{n}{obj\PYZus{}fun}\PY{p}{,} \PY{n}{decoder}\PY{p}{,}
                                  \PY{n}{min\PYZus{}or\PYZus{}max}\PY{o}{=}\PY{n}{MAX}\PY{p}{,} \PY{n}{num\PYZus{}eras}\PY{o}{=}\PY{l+m+mi}{100}\PY{p}{,}
                                  \PY{n}{population\PYZus{}size}\PY{o}{=}\PY{l+m+mi}{20}\PY{p}{,} \PY{n}{chromosome\PYZus{}length}\PY{o}{=}\PY{l+m+mi}{12}\PY{p}{,}
                                  \PY{n}{crossover\PYZus{}probability}\PY{o}{=}\PY{l+m+mf}{0.4}\PY{p}{,}\PY{n}{mutation\PYZus{}probability}\PY{o}{=}\PY{l+m+mf}{0.005}\PY{p}{)}\PY{p}{:}

            \PY{c}{\PYZsh{} define fitness function (decode string, then feed to the OF)}
            \PY{n}{fitness} \PY{o}{=} \PY{k}{lambda} \PY{n}{coding}\PY{p}{:} \PY{n}{obj\PYZus{}fun}\PY{p}{(}\PY{o}{*}\PY{n}{decoder}\PY{p}{(}\PY{n}{coding}\PY{p}{)}\PY{p}{)}

            \PY{c}{\PYZsh{} initialize population}
            \PY{n}{population} \PY{o}{=} \PY{n}{generate\PYZus{}random\PYZus{}population}\PY{p}{(}\PY{n}{number}\PY{o}{=}\PY{n}{population\PYZus{}size}\PY{p}{,}
                                                    \PY{n}{length}\PY{o}{=}\PY{n}{chromosome\PYZus{}length}\PY{p}{)}
            \PY{c}{\PYZsh{} data collection}
            \PY{n}{populations} \PY{o}{=} \PY{p}{[}\PY{n}{population}\PY{p}{]} \PY{c}{\PYZsh{} initialize with first population}

            \PY{c}{\PYZsh{} SGA loop}
            \PY{k}{for} \PY{n}{i} \PY{o+ow}{in} \PY{n+nb}{range}\PY{p}{(}\PY{n}{num\PYZus{}eras}\PY{p}{)}\PY{p}{:}
                \PY{n}{population} \PY{o}{=} \PY{n}{reproduction}\PY{p}{(}\PY{n}{population}\PY{p}{,} \PY{n}{fitness}\PY{p}{,} \PY{n}{min\PYZus{}or\PYZus{}max}\PY{p}{)}
                \PY{n}{population} \PY{o}{=} \PY{n}{population\PYZus{}crossover}\PY{p}{(}\PY{n}{population}\PY{p}{,} \PY{n}{crossover\PYZus{}probability}\PY{p}{)}
                \PY{n}{population} \PY{o}{=} \PY{n}{mutate\PYZus{}population}\PY{p}{(}\PY{n}{population}\PY{p}{,} \PY{n}{mutation\PYZus{}probability}\PY{p}{)}
                \PY{n}{populations}\PY{o}{.}\PY{n}{append}\PY{p}{(}\PY{n}{population}\PY{p}{)} \PY{c}{\PYZsh{} data collection}

            \PY{k}{return} \PY{n}{populations}
\end{Verbatim}

    \section{Testing the SGA
implementation}\label{testing-the-sga-implementation}

\subsection{Benchmark objective
functions}\label{benchmark-objective-functions}

The benchmark objective functions are defined here. They are coding
agnostic, so we focus on delivering an accurate OF implementation and
worry about how to provide its parameters in the accompanying decoder
function (see below).

    \begin{Verbatim}[commandchars=\\\{\}]
{\color{incolor}In [{\color{incolor}7}]:} \PY{k}{def} \PY{n+nf}{dejong\PYZus{}OF}\PY{p}{(}\PY{o}{*}\PY{n}{x}\PY{p}{)}\PY{p}{:}
            \PY{k}{return} \PY{n+nb}{sum}\PY{p}{(}\PY{n}{xi}\PY{o}{*}\PY{o}{*}\PY{l+m+mi}{2} \PY{k}{for} \PY{n}{xi} \PY{o+ow}{in} \PY{n}{x}\PY{p}{)}

        \PY{k}{def} \PY{n+nf}{rosenbrock\PYZus{}OF}\PY{p}{(}\PY{o}{*}\PY{n}{x}\PY{p}{)}\PY{p}{:}
            \PY{n}{irange} \PY{o}{=} \PY{n+nb}{range}\PY{p}{(}\PY{n+nb}{len}\PY{p}{(}\PY{n}{x}\PY{p}{)}\PY{o}{\PYZhy{}}\PY{l+m+mi}{1}\PY{p}{)}
            \PY{k}{return} \PY{n+nb}{sum}\PY{p}{(}\PY{l+m+mi}{100} \PY{o}{*} \PY{p}{(}\PY{n}{x}\PY{p}{[}\PY{n}{i}\PY{o}{+}\PY{l+m+mi}{1}\PY{p}{]} \PY{o}{\PYZhy{}} \PY{n}{x}\PY{p}{[}\PY{n}{i}\PY{p}{]}\PY{o}{*}\PY{o}{*}\PY{l+m+mi}{2}\PY{p}{)}\PY{o}{*}\PY{o}{*}\PY{l+m+mi}{2} \PY{o}{+} \PY{p}{(}\PY{l+m+mi}{1}\PY{o}{\PYZhy{}}\PY{n}{x}\PY{p}{[}\PY{n}{i}\PY{p}{]}\PY{p}{)}\PY{o}{*}\PY{o}{*}\PY{l+m+mi}{2} \PY{k}{for} \PY{n}{i} \PY{o+ow}{in} \PY{n}{irange}\PY{p}{)}

        \PY{k}{def} \PY{n+nf}{himmelblau\PYZus{}OF}\PY{p}{(}\PY{n}{x}\PY{p}{,} \PY{n}{y}\PY{p}{)}\PY{p}{:}
            \PY{k}{return} \PY{p}{(}\PY{n}{x}\PY{o}{*}\PY{o}{*}\PY{l+m+mi}{2} \PY{o}{+} \PY{n}{y} \PY{o}{\PYZhy{}} \PY{l+m+mi}{11}\PY{p}{)}\PY{o}{*}\PY{o}{*}\PY{l+m+mi}{2} \PY{o}{+} \PY{p}{(}\PY{n}{x} \PY{o}{+} \PY{n}{y}\PY{o}{*}\PY{o}{*}\PY{l+m+mi}{2} \PY{o}{\PYZhy{}} \PY{l+m+mi}{7}\PY{p}{)}\PY{o}{*}\PY{o}{*}\PY{l+m+mi}{2}

        \PY{k}{def} \PY{n+nf}{esf}\PY{p}{(}\PY{o}{*}\PY{n}{a}\PY{p}{)}\PY{p}{:}
            \PY{n}{nov} \PY{o}{=} \PY{n+nb}{len}\PY{p}{(}\PY{n}{a}\PY{p}{)} \PY{c}{\PYZsh{} number of variables = length of input vector}
            \PY{n}{terms} \PY{o}{=} \PY{p}{(}\PY{n}{a}\PY{p}{[}\PY{n}{i}\PY{p}{]}\PY{o}{*}\PY{n}{a}\PY{p}{[}\PY{n}{j}\PY{p}{]} \PY{k}{for} \PY{n}{i} \PY{o+ow}{in} \PY{n+nb}{range}\PY{p}{(}\PY{n}{nov}\PY{p}{)} \PY{k}{for} \PY{n}{j} \PY{o+ow}{in} \PY{n+nb}{range}\PY{p}{(}\PY{n}{i}\PY{o}{+}\PY{l+m+mi}{1}\PY{p}{,} \PY{n}{nov}\PY{p}{)}\PY{p}{)}
            \PY{k}{return} \PY{n+nb}{abs}\PY{p}{(}\PY{n+nb}{sum}\PY{p}{(}\PY{n}{terms}\PY{p}{)}\PY{p}{)}
\end{Verbatim}

    \subsection{Decoders}\label{decoders}

As mentioned in the previous section (``Benchmarks''), the objective
functions operate with straightforward parameters. Here we implement
methods that take a binary string and ``decode'' it. These decoder
functions yield parameters which are fed directly into the objective
function implementations above.

    \begin{Verbatim}[commandchars=\\\{\}]
{\color{incolor}In [{\color{incolor}8}]:} \PY{k}{def} \PY{n+nf}{split\PYZus{}string\PYZus{}into\PYZus{}chunks}\PY{p}{(}\PY{n}{string}\PY{p}{,} \PY{n}{n}\PY{p}{)}\PY{p}{:}
            \PY{l+s+sd}{\PYZdq{}\PYZdq{}\PYZdq{}}
        \PY{l+s+sd}{    Helper function.}
        \PY{l+s+sd}{    :param string: [str]}
        \PY{l+s+sd}{    :param n: [int \PYZgt{} 0] chunk size}
        \PY{l+s+sd}{    :returns: List[str] the entire string split into sequential chunks of the }
        \PY{l+s+sd}{        given size (plus the remainder)}
        \PY{l+s+sd}{    }
        \PY{l+s+sd}{    example:}
        \PY{l+s+sd}{    }
        \PY{l+s+sd}{    \PYZgt{}\PYZgt{}\PYZgt{} split\PYZus{}string\PYZus{}into\PYZus{}chunks(\PYZsq{}12345678\PYZsq{}, 3)}
        \PY{l+s+sd}{    [\PYZsq{}123\PYZsq{}, \PYZsq{}456\PYZsq{}, \PYZsq{}78\PYZsq{}]}
        \PY{l+s+sd}{    }
        \PY{l+s+sd}{    \PYZgt{}\PYZgt{}\PYZgt{} split\PYZus{}string\PYZus{}into\PYZus{}chunks(\PYZsq{}12345678\PYZsq{}, 4)}
        \PY{l+s+sd}{    [\PYZsq{}1234\PYZsq{}, \PYZsq{}5678\PYZsq{}]}
        \PY{l+s+sd}{    \PYZdq{}\PYZdq{}\PYZdq{}}
            \PY{k}{return} \PY{p}{[}\PY{n}{string}\PY{p}{[}\PY{n}{i}\PY{p}{:}\PY{n}{i}\PY{o}{+}\PY{n}{n}\PY{p}{]} \PY{k}{for} \PY{n}{i} \PY{o+ow}{in} \PY{n+nb}{range}\PY{p}{(}\PY{l+m+mi}{0}\PY{p}{,} \PY{n+nb}{len}\PY{p}{(}\PY{n}{string}\PY{p}{)}\PY{p}{,} \PY{n}{n}\PY{p}{)}\PY{p}{]}

        \PY{k}{def} \PY{n+nf}{dejong\PYZus{}decoder}\PY{p}{(}\PY{n}{coding}\PY{p}{)}\PY{p}{:}
            \PY{n}{n} \PY{o}{=} \PY{l+m+mi}{4}
            \PY{n}{bits\PYZus{}list} \PY{o}{=} \PY{n}{split\PYZus{}string\PYZus{}into\PYZus{}chunks}\PY{p}{(}\PY{n}{coding}\PY{p}{,} \PY{n}{n}\PY{p}{)}
            \PY{c}{\PYZsh{} take first bit as the sign, and the remaining bits as integers}
            \PY{n}{signs\PYZus{}nums} \PY{o}{=} \PY{p}{[}\PY{p}{(}\PY{o}{\PYZhy{}}\PY{l+m+mi}{1} \PY{k}{if} \PY{n}{bits}\PY{p}{[}\PY{l+m+mi}{0}\PY{p}{]} \PY{o}{==} \PY{l+s}{\PYZsq{}}\PY{l+s}{0}\PY{l+s}{\PYZsq{}} \PY{k}{else} \PY{l+m+mi}{1}\PY{p}{,} \PY{n+nb}{int}\PY{p}{(}\PY{n}{bits}\PY{p}{[}\PY{l+m+mi}{1}\PY{p}{:}\PY{p}{]}\PY{p}{,} \PY{l+m+mi}{2}\PY{p}{)}\PY{p}{)}
                          \PY{k}{for} \PY{n}{bits} \PY{o+ow}{in} \PY{n}{bits\PYZus{}list}\PY{p}{]}
            \PY{c}{\PYZsh{} use modulo to ensure that the numbers fall within the require interval:}
            \PY{c}{\PYZsh{}   \PYZhy{}5.12 ≤ x ≤ 5.12}
            \PY{n}{xlist} \PY{o}{=} \PY{p}{[}\PY{n}{sign} \PY{o}{*} \PY{p}{(}\PY{n}{num} \PY{o}{\PYZpc{}} \PY{l+m+mf}{5.12}\PY{p}{)} \PY{k}{for} \PY{n}{sign}\PY{p}{,} \PY{n}{num} \PY{o+ow}{in} \PY{n}{signs\PYZus{}nums}\PY{p}{]}
            \PY{k}{return} \PY{n}{xlist}

        \PY{k}{def} \PY{n+nf}{rosenbrock\PYZus{}decoder}\PY{p}{(}\PY{n}{coding}\PY{p}{)}\PY{p}{:}
            \PY{n}{n} \PY{o}{=} \PY{l+m+mi}{3}
            \PY{n}{bits\PYZus{}list} \PY{o}{=} \PY{n}{split\PYZus{}string\PYZus{}into\PYZus{}chunks}\PY{p}{(}\PY{n}{coding}\PY{p}{,} \PY{n}{n}\PY{p}{)}
            \PY{c}{\PYZsh{} take first bit as the sign, and the remaining bits as integers}
            \PY{n}{signs\PYZus{}nums} \PY{o}{=} \PY{p}{[}\PY{p}{(}\PY{o}{\PYZhy{}}\PY{l+m+mi}{1} \PY{k}{if} \PY{n}{bits}\PY{p}{[}\PY{l+m+mi}{0}\PY{p}{]} \PY{o}{==} \PY{l+s}{\PYZsq{}}\PY{l+s}{0}\PY{l+s}{\PYZsq{}} \PY{k}{else} \PY{l+m+mi}{1}\PY{p}{,} \PY{n+nb}{int}\PY{p}{(}\PY{n}{bits}\PY{p}{[}\PY{l+m+mi}{1}\PY{p}{:}\PY{p}{]}\PY{p}{,} \PY{l+m+mi}{2}\PY{p}{)}\PY{p}{)}
                          \PY{k}{for} \PY{n}{bits} \PY{o+ow}{in} \PY{n}{bits\PYZus{}list}\PY{p}{]}
            \PY{c}{\PYZsh{} use modulo to ensure that the numbers fall within the require interval:}
            \PY{c}{\PYZsh{}   \PYZhy{}2.048 ≤ x ≤ 2.048}
            \PY{n}{x} \PY{o}{=} \PY{p}{[}\PY{n}{sign} \PY{o}{*} \PY{p}{(}\PY{n}{num} \PY{o}{\PYZpc{}} \PY{l+m+mf}{2.048}\PY{p}{)} \PY{k}{for} \PY{n}{sign}\PY{p}{,} \PY{n}{num} \PY{o+ow}{in} \PY{n}{signs\PYZus{}nums}\PY{p}{]}
            \PY{k}{return} \PY{n}{x}

        \PY{k}{def} \PY{n+nf}{num\PYZus{}in\PYZus{}interval}\PY{p}{(}\PY{n}{lo}\PY{p}{,} \PY{n}{hi}\PY{p}{,} \PY{n}{mult}\PY{p}{,} \PY{n}{steps}\PY{p}{)}\PY{p}{:}
            \PY{l+s+sd}{\PYZdq{}\PYZdq{}\PYZdq{}}
        \PY{l+s+sd}{    Helper function that takes simple parameters to deterministically}
        \PY{l+s+sd}{    yield a floating\PYZhy{}point number in a given interval.}
        \PY{l+s+sd}{    }
        \PY{l+s+sd}{    ex. mult = 6, steps = 10}
        \PY{l+s+sd}{    }
        \PY{l+s+sd}{     |\PYZhy{}\PYZhy{}\PYZhy{}+\PYZhy{}\PYZhy{}\PYZhy{}+\PYZhy{}\PYZhy{}\PYZhy{}+\PYZhy{}\PYZhy{}\PYZhy{}+\PYZhy{}\PYZhy{}\PYZhy{}+\PYZhy{}\PYZhy{}\PYZhy{}|\PYZhy{}\PYZhy{}\PYZhy{}+\PYZhy{}\PYZhy{}\PYZhy{}+\PYZhy{}\PYZhy{}\PYZhy{}+\PYZhy{}\PYZhy{}\PYZhy{}|}
        \PY{l+s+sd}{    lo                      mult             hi}
        \PY{l+s+sd}{    }
        \PY{l+s+sd}{    if low = \PYZhy{}10 and hi = 10, then the result will be }
        \PY{l+s+sd}{      = \PYZhy{}10 + 6*(20/10) }
        \PY{l+s+sd}{      = \PYZhy{}10 + 12}
        \PY{l+s+sd}{      = 2}

        \PY{l+s+sd}{    :param lo: [number] low bound of interval}
        \PY{l+s+sd}{    :param hi: [number] high bound of interval}
        \PY{l+s+sd}{    :param mult: [number ≤ divisor] }
        \PY{l+s+sd}{    :param steps: [int] the number of steps in the interval}
        \PY{l+s+sd}{    :returns: [float] a number between `lo` and `hi`}
        \PY{l+s+sd}{    \PYZdq{}\PYZdq{}\PYZdq{}}
            \PY{n}{step\PYZus{}size} \PY{o}{=} \PY{p}{(}\PY{n}{hi} \PY{o}{\PYZhy{}} \PY{n}{lo}\PY{p}{)}\PY{o}{/}\PY{n}{steps}
            \PY{k}{return} \PY{n}{lo} \PY{o}{+} \PY{n}{mult}\PY{o}{*}\PY{n}{step\PYZus{}size}

        \PY{k}{def} \PY{n+nf}{himmelblau\PYZus{}decoder}\PY{p}{(}\PY{n}{coding}\PY{p}{)}\PY{p}{:}
            \PY{n}{mid} \PY{o}{=} \PY{n+nb}{int}\PY{p}{(}\PY{n+nb}{len}\PY{p}{(}\PY{n}{coding}\PY{p}{)}\PY{o}{/}\PY{l+m+mi}{2}\PY{p}{)}
            \PY{c}{\PYZsh{} split string into x param and y param}
            \PY{n}{binx}\PY{p}{,} \PY{n}{biny} \PY{o}{=} \PY{n}{coding}\PY{p}{[}\PY{p}{:}\PY{n}{mid}\PY{p}{]}\PY{p}{,} \PY{n}{coding}\PY{p}{[}\PY{n}{mid}\PY{p}{:}\PY{p}{]}
            \PY{c}{\PYZsh{} use binary x and y as interval multiplier}
            \PY{n}{xmult}\PY{p}{,} \PY{n}{ymult} \PY{o}{=} \PY{n+nb}{int}\PY{p}{(}\PY{n}{binx}\PY{p}{,} \PY{l+m+mi}{2}\PY{p}{)}\PY{p}{,} \PY{n+nb}{int}\PY{p}{(}\PY{n}{biny}\PY{p}{,} \PY{l+m+mi}{2}\PY{p}{)}
            \PY{c}{\PYZsh{} the divisor is the highest possible value x or y could be}
            \PY{c}{\PYZsh{} which is 2**\PYZob{}length of binary string encoding x or y\PYZcb{}}
            \PY{n}{x} \PY{o}{=} \PY{n}{num\PYZus{}in\PYZus{}interval}\PY{p}{(}\PY{o}{\PYZhy{}}\PY{l+m+mi}{4}\PY{p}{,} \PY{l+m+mi}{4}\PY{p}{,} \PY{n}{xmult}\PY{p}{,} \PY{l+m+mi}{2}\PY{o}{*}\PY{o}{*}\PY{n+nb}{len}\PY{p}{(}\PY{n}{binx}\PY{p}{)}\PY{p}{)}
            \PY{n}{y} \PY{o}{=} \PY{n}{num\PYZus{}in\PYZus{}interval}\PY{p}{(}\PY{o}{\PYZhy{}}\PY{l+m+mi}{4}\PY{p}{,} \PY{l+m+mi}{4}\PY{p}{,} \PY{n}{ymult}\PY{p}{,} \PY{l+m+mi}{2}\PY{o}{*}\PY{o}{*}\PY{n+nb}{len}\PY{p}{(}\PY{n}{biny}\PY{p}{)}\PY{p}{)}
            \PY{k}{return} \PY{n}{x}\PY{p}{,} \PY{n}{y}

        \PY{k}{def} \PY{n+nf}{esf\PYZus{}decoder}\PY{p}{(}\PY{n}{coding}\PY{p}{)}\PY{p}{:}
            \PY{k}{return} \PY{p}{[}\PY{o}{\PYZhy{}}\PY{l+m+mi}{1} \PY{k}{if} \PY{n}{char} \PY{o}{==} \PY{l+s}{\PYZsq{}}\PY{l+s}{0}\PY{l+s}{\PYZsq{}} \PY{k}{else} \PY{l+m+mi}{1} \PY{k}{for} \PY{n}{char} \PY{o+ow}{in} \PY{n}{coding}\PY{p}{]}
\end{Verbatim}

    \subsection{Tools for plotting}\label{tools-for-plotting}

    \begin{Verbatim}[commandchars=\\\{\}]
{\color{incolor}In [{\color{incolor}9}]:} \PY{k}{def} \PY{n+nf}{plot\PYZus{}ga}\PY{p}{(}\PY{n}{obj\PYZus{}fun}\PY{p}{,} \PY{n}{decoder}\PY{p}{,} \PY{n}{ax}\PY{o}{=}\PY{k}{None}\PY{p}{,} \PY{n}{ga\PYZus{}opts}\PY{o}{=}\PY{k}{None}\PY{p}{,} \PY{n}{min\PYZus{}or\PYZus{}max}\PY{o}{=}\PY{n}{MIN}\PY{p}{,}
                    \PY{n}{title}\PY{o}{=}\PY{l+s}{\PYZdq{}}\PY{l+s}{Genetic Algorithm Evolution}\PY{l+s}{\PYZdq{}}\PY{p}{,} \PY{n}{legend}\PY{o}{=}\PY{k}{True}\PY{p}{)}\PY{p}{:}
            \PY{k}{if} \PY{n}{ga\PYZus{}opts} \PY{o+ow}{is} \PY{k}{None}\PY{p}{:}
                \PY{n}{ga\PYZus{}opts} \PY{o}{=} \PY{p}{\PYZob{}}\PY{p}{\PYZcb{}}

            \PY{n}{ga\PYZus{}opts}\PY{p}{[}\PY{l+s}{\PYZsq{}}\PY{l+s}{min\PYZus{}or\PYZus{}max}\PY{l+s}{\PYZsq{}}\PY{p}{]} \PY{o}{=} \PY{n}{min\PYZus{}or\PYZus{}max}
            \PY{c}{\PYZsh{} run SGA}
            \PY{n}{populations} \PY{o}{=} \PY{n}{run\PYZus{}genetic\PYZus{}algorithm}\PY{p}{(}\PY{n}{obj\PYZus{}fun}\PY{p}{,} \PY{n}{decoder}\PY{p}{,} \PY{o}{*}\PY{o}{*}\PY{n}{ga\PYZus{}opts}\PY{o}{.}\PY{n}{copy}\PY{p}{(}\PY{p}{)}\PY{p}{)}

            \PY{c}{\PYZsh{} define fitness func}
            \PY{n}{fitness} \PY{o}{=} \PY{k}{lambda} \PY{n}{c}\PY{p}{:} \PY{n}{obj\PYZus{}fun}\PY{p}{(}\PY{o}{*}\PY{n}{decoder}\PY{p}{(}\PY{n}{c}\PY{p}{)}\PY{p}{)}

            \PY{c}{\PYZsh{} Find the \PYZdq{}global optimum\PYZdq{} of all the chromosomes we looked at.}
            \PY{c}{\PYZsh{} A better term for this chromosome is \PYZdq{}best individual\PYZdq{}.}
            \PY{n}{all\PYZus{}chromosomes} \PY{o}{=} \PY{p}{\PYZob{}}\PY{n}{c} \PY{k}{for} \PY{n}{pop} \PY{o+ow}{in} \PY{n}{populations} \PY{k}{for} \PY{n}{c} \PY{o+ow}{in} \PY{n}{pop}\PY{p}{\PYZcb{}}
            \PY{n}{optimizer} \PY{o}{=} \PY{n+nb}{min} \PY{k}{if} \PY{n}{min\PYZus{}or\PYZus{}max} \PY{o}{==} \PY{n}{MIN} \PY{k}{else} \PY{n+nb}{max}
            \PY{n}{global\PYZus{}optimum} \PY{o}{=} \PY{n}{optimizer}\PY{p}{(}\PY{n}{all\PYZus{}chromosomes}\PY{p}{,} \PY{n}{key}\PY{o}{=}\PY{n}{fitness}\PY{p}{)}
            \PY{n}{fittest\PYZus{}fitness} \PY{o}{=} \PY{n}{fitness}\PY{p}{(}\PY{n}{global\PYZus{}optimum}\PY{p}{)}

            \PY{c}{\PYZsh{} Print the optimum to the console}
            \PY{n+nb}{print}\PY{p}{(}\PY{l+s}{\PYZdq{}}\PY{l+s}{Global optimum:}\PY{l+s}{\PYZdq{}}\PY{p}{,} \PY{n}{global\PYZus{}optimum}\PY{p}{)}
            \PY{n+nb}{print}\PY{p}{(}\PY{l+s}{\PYZdq{}}\PY{l+s}{Fitness:}\PY{l+s}{\PYZdq{}}\PY{p}{,} \PY{n}{fittest\PYZus{}fitness}\PY{p}{)}
            \PY{n+nb}{print}\PY{p}{(}\PY{l+s}{\PYZdq{}}\PY{l+s}{Decoded:}\PY{l+s}{\PYZdq{}}\PY{p}{,} \PY{n}{decoder}\PY{p}{(}\PY{n}{global\PYZus{}optimum}\PY{p}{)}\PY{p}{)}

            \PY{c}{\PYZsh{} Start plotting}
            \PY{c}{\PYZsh{} Define the data ranges}
            \PY{n}{x\PYZus{}axis} \PY{o}{=} \PY{n+nb}{range}\PY{p}{(}\PY{n+nb}{len}\PY{p}{(}\PY{n}{populations}\PY{p}{)}\PY{p}{)}
            \PY{n}{fitnesses} \PY{o}{=} \PY{p}{[}\PY{p}{[}\PY{n}{fitness}\PY{p}{(}\PY{n}{m}\PY{p}{)} \PY{k}{for} \PY{n}{m} \PY{o+ow}{in} \PY{n}{population}\PY{p}{]} \PY{k}{for} \PY{n}{population} \PY{o+ow}{in} \PY{n}{populations}\PY{p}{]}
            \PY{n}{mins} \PY{o}{=} \PY{p}{[}\PY{n+nb}{min}\PY{p}{(}\PY{n}{f}\PY{p}{)} \PY{k}{for} \PY{n}{f} \PY{o+ow}{in} \PY{n}{fitnesses}\PY{p}{]}
            \PY{n}{maxs} \PY{o}{=} \PY{p}{[}\PY{n+nb}{max}\PY{p}{(}\PY{n}{f}\PY{p}{)} \PY{k}{for} \PY{n}{f} \PY{o+ow}{in} \PY{n}{fitnesses}\PY{p}{]}
            \PY{n}{avgs} \PY{o}{=} \PY{p}{[}\PY{n+nb}{sum}\PY{p}{(}\PY{n}{f}\PY{p}{)}\PY{o}{/}\PY{n+nb}{len}\PY{p}{(}\PY{n}{f}\PY{p}{)} \PY{k}{for} \PY{n}{f} \PY{o+ow}{in} \PY{n}{fitnesses}\PY{p}{]}
            \PY{n}{optima} \PY{o}{=} \PY{p}{[}\PY{p}{(}\PY{n}{it}\PY{p}{,} \PY{n}{fittest\PYZus{}fitness}\PY{p}{)} \PY{k}{for} \PY{n}{it}\PY{p}{,} \PY{n}{pop} \PY{o+ow}{in} \PY{n+nb}{enumerate}\PY{p}{(}\PY{n}{populations}\PY{p}{)}
                      \PY{k}{if} \PY{n}{fittest\PYZus{}fitness} \PY{o+ow}{in} \PY{n+nb}{map}\PY{p}{(}\PY{n}{fitness}\PY{p}{,} \PY{n}{pop}\PY{p}{)}\PY{p}{]}
            \PY{n}{x\PYZus{}optima}\PY{p}{,} \PY{n}{y\PYZus{}optima} \PY{o}{=} \PY{n+nb}{zip}\PY{p}{(}\PY{o}{*}\PY{n}{optima}\PY{p}{)} \PY{c}{\PYZsh{} unzip pairs into two sequences}

            \PY{k}{if} \PY{n}{ax} \PY{o+ow}{is} \PY{k}{None}\PY{p}{:} \PY{c}{\PYZsh{} if no plotting axes are provided}
                \PY{c}{\PYZsh{} define a set of axes}
                \PY{n}{fig}\PY{p}{,} \PY{n}{ax} \PY{o}{=} \PY{n}{plt}\PY{o}{.}\PY{n}{subplots}\PY{p}{(}\PY{l+m+mi}{1}\PY{p}{)}

            \PY{c}{\PYZsh{} do the plotting}
            \PY{n}{l\PYZus{}mins}\PY{p}{,} \PY{n}{l\PYZus{}maxs}\PY{p}{,} \PY{n}{l\PYZus{}avgs} \PY{o}{=} \PY{n}{ax}\PY{o}{.}\PY{n}{plot}\PY{p}{(}\PY{n}{x\PYZus{}axis}\PY{p}{,} \PY{n}{mins}\PY{p}{,} \PY{l+s}{\PYZsq{}}\PY{l+s}{r\PYZhy{}\PYZhy{}}\PY{l+s}{\PYZsq{}}\PY{p}{,} \PY{n}{maxs}\PY{p}{,} \PY{l+s}{\PYZsq{}}\PY{l+s}{b\PYZhy{}\PYZhy{}}\PY{l+s}{\PYZsq{}}\PY{p}{,} \PY{n}{avgs}\PY{p}{,} \PY{l+s}{\PYZsq{}}\PY{l+s}{g\PYZhy{}}\PY{l+s}{\PYZsq{}}\PY{p}{)}
            \PY{n}{scatter\PYZus{}ceil} \PY{o}{=} \PY{n}{ax}\PY{o}{.}\PY{n}{scatter}\PY{p}{(}\PY{n}{x\PYZus{}optima}\PY{p}{,} \PY{n}{y\PYZus{}optima}\PY{p}{,} \PY{n}{c}\PY{o}{=}\PY{l+s}{\PYZsq{}}\PY{l+s}{purple}\PY{l+s}{\PYZsq{}}\PY{p}{)}

            \PY{c}{\PYZsh{} create a legend}
            \PY{k}{if} \PY{n}{legend}\PY{p}{:}
                \PY{n}{plt}\PY{o}{.}\PY{n}{legend}\PY{p}{(}
                    \PY{p}{(}\PY{n}{l\PYZus{}mins}\PY{p}{,} \PY{n}{l\PYZus{}maxs}\PY{p}{,} \PY{n}{l\PYZus{}avgs}\PY{p}{,} \PY{n}{scatter\PYZus{}ceil}\PY{p}{)}\PY{p}{,}
                    \PY{p}{(}\PY{l+s}{\PYZdq{}}\PY{l+s}{min pop fitness}\PY{l+s}{\PYZdq{}}\PY{p}{,} \PY{l+s}{\PYZdq{}}\PY{l+s}{max pop fitness}\PY{l+s}{\PYZdq{}}\PY{p}{,} \PY{l+s}{\PYZdq{}}\PY{l+s}{average pop fitness}\PY{l+s}{\PYZdq{}}\PY{p}{,}
                     \PY{l+s}{\PYZdq{}}\PY{l+s}{occurrences of global optimum}\PY{l+s}{\PYZdq{}}\PY{p}{)}\PY{p}{,}
                    \PY{n}{loc}\PY{o}{=}\PY{l+s}{\PYZdq{}}\PY{l+s}{upper right}\PY{l+s}{\PYZdq{}}\PY{p}{,}
                \PY{p}{)}

            \PY{c}{\PYZsh{} set parameters for the axes}
            \PY{n}{ax}\PY{o}{.}\PY{n}{set\PYZus{}xlim}\PY{p}{(}\PY{l+m+mi}{0}\PY{p}{,} \PY{n+nb}{len}\PY{p}{(}\PY{n}{populations}\PY{p}{)}\PY{p}{)}
            \PY{n}{ax}\PY{o}{.}\PY{n}{set\PYZus{}ylim}\PY{p}{(}\PY{l+m+mi}{0}\PY{p}{,} \PY{n+nb}{int}\PY{p}{(}\PY{n+nb}{max}\PY{p}{(}\PY{n}{maxs}\PY{p}{)} \PY{o}{*} \PY{l+m+mf}{1.20}\PY{p}{)}\PY{p}{)}
            \PY{n}{ax}\PY{o}{.}\PY{n}{set\PYZus{}title}\PY{p}{(}\PY{n}{title}\PY{p}{)}
            \PY{n}{ax}\PY{o}{.}\PY{n}{set\PYZus{}xlabel}\PY{p}{(}\PY{l+s}{\PYZdq{}}\PY{l+s}{era}\PY{l+s}{\PYZdq{}}\PY{p}{)}
            \PY{n}{ax}\PY{o}{.}\PY{n}{set\PYZus{}ylabel}\PY{p}{(}\PY{l+s}{\PYZdq{}}\PY{l+s}{fitness}\PY{l+s}{\PYZdq{}}\PY{p}{)}

            \PY{k}{return} \PY{n}{ax}
\end{Verbatim}

    \begin{Verbatim}[commandchars=\\\{\}]
{\color{incolor}In [{\color{incolor}10}]:} \PY{k}{def} \PY{n+nf}{plot\PYZus{}ga\PYZus{}minmax}\PY{p}{(}\PY{n}{objfun}\PY{p}{,} \PY{n}{decoder}\PY{p}{,} \PY{n}{min\PYZus{}ga\PYZus{}opts}\PY{o}{=}\PY{k}{None}\PY{p}{,} \PY{n}{max\PYZus{}ga\PYZus{}opts}\PY{o}{=}\PY{k}{None}\PY{p}{,}
                            \PY{n}{title}\PY{o}{=}\PY{l+s}{\PYZdq{}}\PY{l+s}{Genetic Algorithm}\PY{l+s}{\PYZdq{}}\PY{p}{)}\PY{p}{:}
             \PY{n}{fig}\PY{p}{,} \PY{p}{(}\PY{n}{ax1}\PY{p}{,} \PY{n}{ax2}\PY{p}{)} \PY{o}{=} \PY{n}{plt}\PY{o}{.}\PY{n}{subplots}\PY{p}{(}\PY{l+m+mi}{1}\PY{p}{,} \PY{l+m+mi}{2}\PY{p}{)}
             \PY{k}{for} \PY{n}{ax}\PY{p}{,} \PY{n}{minimax}\PY{p}{,} \PY{n}{opts} \PY{o+ow}{in} \PY{p}{[}\PY{p}{(}\PY{n}{ax1}\PY{p}{,} \PY{n}{MIN}\PY{p}{,} \PY{n}{min\PYZus{}ga\PYZus{}opts}\PY{p}{)}\PY{p}{,} \PY{p}{(}\PY{n}{ax2}\PY{p}{,} \PY{n}{MAX}\PY{p}{,} \PY{n}{max\PYZus{}ga\PYZus{}opts}\PY{p}{)}\PY{p}{]}\PY{p}{:}
                 \PY{n}{opts} \PY{o}{=} \PY{n}{opts} \PY{o+ow}{or} \PY{p}{\PYZob{}}\PY{p}{\PYZcb{}} \PY{c}{\PYZsh{} if none, then use empty dict of options}
                 \PY{n}{minimax\PYZus{}title} \PY{o}{=} \PY{l+s}{\PYZdq{}}\PY{l+s}{minimization}\PY{l+s}{\PYZdq{}} \PY{k}{if} \PY{n}{minimax} \PY{o}{==} \PY{n}{MIN} \PY{k}{else} \PY{l+s}{\PYZdq{}}\PY{l+s}{maximization}\PY{l+s}{\PYZdq{}}
                 \PY{n+nb}{print}\PY{p}{(}\PY{l+s}{\PYZsq{}}\PY{l+s+se}{\PYZbs{}n}\PY{l+s}{\PYZsq{}}\PY{p}{,} \PY{n}{minimax\PYZus{}title}\PY{p}{,} \PY{l+s}{\PYZsq{}}\PY{l+s+se}{\PYZbs{}n}\PY{l+s}{\PYZsq{}}\PY{p}{,} \PY{l+s}{\PYZsq{}}\PY{l+s}{=}\PY{l+s}{\PYZsq{}}\PY{o}{*}\PY{n+nb}{len}\PY{p}{(}\PY{n}{minimax\PYZus{}title}\PY{p}{)}\PY{p}{,} \PY{l+s}{\PYZsq{}}\PY{l+s+se}{\PYZbs{}n}\PY{l+s}{\PYZsq{}}\PY{p}{)}
                 \PY{n}{plot\PYZus{}ga}\PY{p}{(}
                     \PY{n}{objfun}\PY{p}{,} \PY{n}{decoder}\PY{p}{,}
                     \PY{n}{ax}\PY{o}{=}\PY{n}{ax}\PY{p}{,} \PY{n}{min\PYZus{}or\PYZus{}max}\PY{o}{=}\PY{n}{minimax}\PY{p}{,}
                     \PY{n}{ga\PYZus{}opts}\PY{o}{=}\PY{n}{opts}\PY{p}{,}
                     \PY{n}{title}\PY{o}{=}\PY{l+s}{\PYZdq{}}\PY{l+s}{\PYZob{}\PYZcb{} (\PYZob{}\PYZcb{})}\PY{l+s}{\PYZdq{}}\PY{o}{.}\PY{n}{format}\PY{p}{(}\PY{n}{title}\PY{p}{,} \PY{n}{minimax\PYZus{}title}\PY{p}{)}\PY{p}{,}
                     \PY{n}{legend}\PY{o}{=}\PY{k}{False} \PY{c}{\PYZsh{} no space on the min/max graphs}
                 \PY{p}{)}


         \PY{k}{def} \PY{n+nf}{plot\PYZus{}esf\PYZus{}minmax}\PY{p}{(}\PY{n}{nov}\PY{p}{,} \PY{n}{min\PYZus{}ga\PYZus{}opts}\PY{o}{=}\PY{k}{None}\PY{p}{,} \PY{n}{max\PYZus{}ga\PYZus{}opts}\PY{o}{=}\PY{k}{None}\PY{p}{)}\PY{p}{:}
             \PY{c}{\PYZsh{} initialize option dicts as new dictionaries}
             \PY{n}{min\PYZus{}opts} \PY{o}{=} \PY{p}{\PYZob{}}\PY{p}{\PYZcb{}}
             \PY{n}{max\PYZus{}opts} \PY{o}{=} \PY{p}{\PYZob{}}\PY{p}{\PYZcb{}}
             \PY{c}{\PYZsh{} update with parameter option dicts (if provided)}
             \PY{n}{min\PYZus{}opts}\PY{o}{.}\PY{n}{update}\PY{p}{(}\PY{n}{min\PYZus{}ga\PYZus{}opts} \PY{o+ow}{or} \PY{p}{\PYZob{}}\PY{p}{\PYZcb{}}\PY{p}{)}
             \PY{n}{max\PYZus{}opts}\PY{o}{.}\PY{n}{update}\PY{p}{(}\PY{n}{max\PYZus{}ga\PYZus{}opts} \PY{o+ow}{or} \PY{p}{\PYZob{}}\PY{p}{\PYZcb{}}\PY{p}{)}
             \PY{c}{\PYZsh{} update with mandatory chromosome length = NOV}
             \PY{n}{min\PYZus{}opts}\PY{p}{[}\PY{l+s}{\PYZsq{}}\PY{l+s}{chromosome\PYZus{}length}\PY{l+s}{\PYZsq{}}\PY{p}{]} \PY{o}{=} \PY{n}{nov}
             \PY{n}{max\PYZus{}opts}\PY{p}{[}\PY{l+s}{\PYZsq{}}\PY{l+s}{chromosome\PYZus{}length}\PY{l+s}{\PYZsq{}}\PY{p}{]} \PY{o}{=} \PY{n}{nov}
             \PY{c}{\PYZsh{} plot!}
             \PY{n}{plot\PYZus{}ga\PYZus{}minmax}\PY{p}{(}\PY{n}{esf}\PY{p}{,} \PY{n}{esf\PYZus{}decoder}\PY{p}{,} \PY{n}{title}\PY{o}{=}\PY{l+s}{\PYZdq{}}\PY{l+s}{ESF (\PYZob{}\PYZcb{} vars)}\PY{l+s}{\PYZdq{}}\PY{o}{.}\PY{n}{format}\PY{p}{(}\PY{n}{nov}\PY{p}{)}\PY{p}{,}
                            \PY{n}{min\PYZus{}ga\PYZus{}opts}\PY{o}{=}\PY{n}{min\PYZus{}opts}\PY{o}{.}\PY{n}{copy}\PY{p}{(}\PY{p}{)}\PY{p}{,} \PY{n}{max\PYZus{}ga\PYZus{}opts}\PY{o}{=}\PY{n}{max\PYZus{}opts}\PY{o}{.}\PY{n}{copy}\PY{p}{(}\PY{p}{)}
             \PY{p}{)}
\end{Verbatim}

    \section{Benchmark Performance}\label{benchmark-performance}

    \subsection{DeJong (Sphere) Function}\label{dejong-sphere-function}

Optimal where all values \(x_i = 0\).

This converges pretty quickly so I only use 50 eras of evolution.

    \begin{Verbatim}[commandchars=\\\{\}]
{\color{incolor}In [{\color{incolor}11}]:} \PY{n}{decoder} \PY{o}{=} \PY{n}{dejong\PYZus{}decoder}
         \PY{n}{obj\PYZus{}fun} \PY{o}{=} \PY{n}{dejong\PYZus{}OF}

         \PY{n}{ga\PYZus{}options} \PY{o}{=} \PY{n+nb}{dict}\PY{p}{(}
             \PY{n}{num\PYZus{}eras}\PY{o}{=}\PY{l+m+mi}{100}\PY{p}{,} \PY{n}{population\PYZus{}size}\PY{o}{=}\PY{l+m+mi}{40}\PY{p}{,} \PY{n}{chromosome\PYZus{}length}\PY{o}{=}\PY{l+m+mi}{20}\PY{p}{,}
             \PY{n}{crossover\PYZus{}probability}\PY{o}{=}\PY{l+m+mf}{0.3}\PY{p}{,} \PY{n}{mutation\PYZus{}probability}\PY{o}{=}\PY{l+m+mf}{0.05}
         \PY{p}{)}

         \PY{n}{plot\PYZus{}ga}\PY{p}{(}\PY{n}{obj\PYZus{}fun}\PY{p}{,} \PY{n}{decoder}\PY{p}{,} \PY{n}{min\PYZus{}or\PYZus{}max}\PY{o}{=}\PY{n}{MIN}\PY{p}{,}
                 \PY{n}{ga\PYZus{}opts}\PY{o}{=}\PY{n}{ga\PYZus{}options}\PY{p}{,} \PY{n}{title}\PY{o}{=}\PY{l+s}{\PYZdq{}}\PY{l+s}{DeJong Function (minimization)}\PY{l+s}{\PYZdq{}}\PY{p}{)}
\end{Verbatim}

    \begin{Verbatim}[commandchars=\\\{\}]
Global optimum: 00000000000000000000
Fitness: 0.0
Decoded: [-0.0, -0.0, -0.0, -0.0, -0.0]
    \end{Verbatim}

            \begin{Verbatim}[commandchars=\\\{\}]
{\color{outcolor}Out[{\color{outcolor}11}]:} <matplotlib.axes.\_subplots.AxesSubplot at 0x112cbacc0>
\end{Verbatim}

    \begin{center}
    \adjustimage{max size={0.9\linewidth}{0.9\paperheight}}{SGA_files/SGA_23_2.png}
    \end{center}
    { \hspace*{\fill} \\}

    \subsection{Rosenbrock Function}\label{rosenbrock-function}

    Optimal where all values \(x_i = 1\).  This does better when more mutation is allowed and
    there are a few more eras.

    \begin{Verbatim}[commandchars=\\\{\}]
{\color{incolor}In [{\color{incolor}12}]:} \PY{n}{decoder} \PY{o}{=} \PY{n}{rosenbrock\PYZus{}decoder}
         \PY{n}{obj\PYZus{}fun} \PY{o}{=} \PY{n}{rosenbrock\PYZus{}OF}

         \PY{n}{ga\PYZus{}options} \PY{o}{=} \PY{n+nb}{dict}\PY{p}{(}
             \PY{n}{num\PYZus{}eras}\PY{o}{=}\PY{l+m+mi}{100}\PY{p}{,} \PY{n}{population\PYZus{}size}\PY{o}{=}\PY{l+m+mi}{40}\PY{p}{,} \PY{n}{chromosome\PYZus{}length}\PY{o}{=}\PY{l+m+mi}{20}\PY{p}{,}
             \PY{n}{crossover\PYZus{}probability}\PY{o}{=}\PY{l+m+mf}{0.35}\PY{p}{,} \PY{n}{mutation\PYZus{}probability}\PY{o}{=}\PY{l+m+mf}{0.04}
         \PY{p}{)}

         \PY{n}{plot\PYZus{}ga}\PY{p}{(}\PY{n}{obj\PYZus{}fun}\PY{p}{,} \PY{n}{decoder}\PY{p}{,} \PY{n}{min\PYZus{}or\PYZus{}max}\PY{o}{=}\PY{n}{MIN}\PY{p}{,} \PY{n}{ga\PYZus{}opts}\PY{o}{=}\PY{n}{ga\PYZus{}options}\PY{p}{,}
                 \PY{n}{title}\PY{o}{=}\PY{l+s}{\PYZdq{}}\PY{l+s}{Rosenbrock Function (minimization)}\PY{l+s}{\PYZdq{}}\PY{p}{)}
\end{Verbatim}

    \begin{Verbatim}[commandchars=\\\{\}]
Global optimum: 10110110110110110111
Fitness: 0.0
Decoded: [1.0, 1.0, 1.0, 1.0, 1.0, 1.0, 1.0]
    \end{Verbatim}

            \begin{Verbatim}[commandchars=\\\{\}]
{\color{outcolor}Out[{\color{outcolor}12}]:} <matplotlib.axes.\_subplots.AxesSubplot at 0x1167b7710>
\end{Verbatim}

    \begin{center}
    \adjustimage{max size={0.9\linewidth}{0.9\paperheight}}{SGA_files/SGA_25_2.png}
    \end{center}
    { \hspace*{\fill} \\}

    \subsection{Himmelblau's Function}\label{himmelblaus-function}

It has four identical local minima:

\begin{itemize}
\tightlist
\item
  \(f(3.0, 2.0) = 0.0\),
\item
  \(f(-2.805118, 3.131312) = 0.0\),
\item
  \(f(-3.779310, -3.283186) = 0.0\),
\item
  \(f(3.584428, -1.848126) = 0.0\)
\end{itemize}

We work with a larger chromosome length to get higher precision real
numbers.

    \begin{Verbatim}[commandchars=\\\{\}]
{\color{incolor}In [{\color{incolor}13}]:} \PY{n}{decoder} \PY{o}{=} \PY{n}{himmelblau\PYZus{}decoder}
         \PY{n}{obj\PYZus{}fun} \PY{o}{=} \PY{n}{himmelblau\PYZus{}OF}

         \PY{n}{ga\PYZus{}opts} \PY{o}{=} \PY{n+nb}{dict}\PY{p}{(}
             \PY{n}{num\PYZus{}eras}\PY{o}{=}\PY{l+m+mi}{100}\PY{p}{,} \PY{n}{population\PYZus{}size}\PY{o}{=}\PY{l+m+mi}{40}\PY{p}{,} \PY{n}{chromosome\PYZus{}length}\PY{o}{=}\PY{l+m+mi}{64}\PY{p}{,}
             \PY{n}{crossover\PYZus{}probability}\PY{o}{=}\PY{l+m+mf}{0.35}\PY{p}{,} \PY{n}{mutation\PYZus{}probability}\PY{o}{=}\PY{l+m+mf}{0.04}
         \PY{p}{)}

         \PY{n}{plot\PYZus{}ga}\PY{p}{(}\PY{n}{obj\PYZus{}fun}\PY{p}{,} \PY{n}{decoder}\PY{p}{,} \PY{n}{min\PYZus{}or\PYZus{}max}\PY{o}{=}\PY{n}{MIN}\PY{p}{,} \PY{n}{ga\PYZus{}opts}\PY{o}{=}\PY{n}{ga\PYZus{}opts}\PY{p}{,}
                       \PY{n}{title}\PY{o}{=}\PY{l+s}{\PYZdq{}}\PY{l+s}{Himmelblau}\PY{l+s}{\PYZsq{}}\PY{l+s}{s Function}\PY{l+s}{\PYZdq{}}\PY{p}{)}
\end{Verbatim}

    \begin{Verbatim}[commandchars=\\\{\}]
Global optimum: 1111001010111101110111110101110101000100110111010000101110101111
Fitness: 8.294811664014935e-05
Decoded: (3.585677796974778, -1.8480168897658587)
    \end{Verbatim}

            \begin{Verbatim}[commandchars=\\\{\}]
{\color{outcolor}Out[{\color{outcolor}13}]:} <matplotlib.axes.\_subplots.AxesSubplot at 0x11347c748>
\end{Verbatim}

    \begin{center}
    \adjustimage{max size={0.9\linewidth}{0.9\paperheight}}{SGA_files/SGA_27_2.png}
    \end{center}
    { \hspace*{\fill} \\}

    \subsection{ESF benchmarks}\label{esf-benchmarks}

Here we test the SGA performance on the elementary symmetric function.

    \subsubsection{Simple brute force checker for minima and
maxima}\label{simple-brute-force-checker-for-minima-and-maxima}

This small routine generates all binary strings of length \texttt{nov}
and computes the fitness for all of those strings.

This is a simple way to find the global optima for the ESF benchmarks,
but it is slow for more than 15 variables or so since there are
\(2^{\text{nov}}\) binary strings length \texttt{nov}.

    \begin{Verbatim}[commandchars=\\\{\}]
{\color{incolor}In [{\color{incolor}14}]:} \PY{k}{def} \PY{n+nf}{solve\PYZus{}esf}\PY{p}{(}\PY{n}{nov}\PY{p}{,} \PY{n}{fitness}\PY{p}{)}\PY{p}{:}
             \PY{n}{all\PYZus{}binstrings} \PY{o}{=} \PY{p}{[}\PY{l+s}{\PYZdq{}}\PY{l+s}{\PYZob{}:0\PYZob{}width\PYZcb{}b\PYZcb{}}\PY{l+s}{\PYZdq{}}\PY{o}{.}\PY{n}{format}\PY{p}{(}\PY{n}{i}\PY{p}{,} \PY{n}{width}\PY{o}{=}\PY{n}{nov}\PY{p}{)} \PY{k}{for} \PY{n}{i} \PY{o+ow}{in} \PY{n+nb}{range}\PY{p}{(}\PY{l+m+mi}{2}\PY{o}{*}\PY{o}{*}\PY{n}{nov}\PY{p}{)}\PY{p}{]}
             \PY{n}{the\PYZus{}min} \PY{o}{=} \PY{n+nb}{min}\PY{p}{(}\PY{n}{all\PYZus{}binstrings}\PY{p}{,} \PY{n}{key}\PY{o}{=}\PY{n}{fitness}\PY{p}{)}
             \PY{n}{the\PYZus{}max} \PY{o}{=} \PY{n+nb}{max}\PY{p}{(}\PY{n}{all\PYZus{}binstrings}\PY{p}{,} \PY{n}{key}\PY{o}{=}\PY{n}{fitness}\PY{p}{)}
             \PY{k}{return} \PY{p}{(}\PY{n}{the\PYZus{}min}\PY{p}{,} \PY{n}{the\PYZus{}max}\PY{p}{)}

         \PY{k}{def} \PY{n+nf}{solve\PYZus{}and\PYZus{}print}\PY{p}{(}\PY{n}{nov}\PY{p}{)}\PY{p}{:}
             \PY{n}{fitness} \PY{o}{=} \PY{k}{lambda} \PY{n}{binstring}\PY{p}{:} \PY{n}{esf}\PY{p}{(}\PY{o}{*}\PY{n}{esf\PYZus{}decoder}\PY{p}{(}\PY{n}{binstring}\PY{p}{)}\PY{p}{)}
             \PY{n}{the\PYZus{}min}\PY{p}{,} \PY{n}{the\PYZus{}max} \PY{o}{=} \PY{n}{solve\PYZus{}esf}\PY{p}{(}\PY{n}{nov}\PY{p}{,} \PY{n}{fitness}\PY{p}{)}
             \PY{n+nb}{print}\PY{p}{(}\PY{l+s}{\PYZsq{}}\PY{l+s}{nov\PYZob{}\PYZcb{}  =\PYZgt{}  min fitness = \PYZob{}:\PYZlt{}4\PYZcb{}  max fitness = \PYZob{}:\PYZlt{}4\PYZcb{}}\PY{l+s}{\PYZsq{}}
                   \PY{o}{.}\PY{n}{format}\PY{p}{(}\PY{n}{nov}\PY{p}{,} \PY{n}{fitness}\PY{p}{(}\PY{n}{the\PYZus{}min}\PY{p}{)}\PY{p}{,} \PY{n}{fitness}\PY{p}{(}\PY{n}{the\PYZus{}max}\PY{p}{)}\PY{p}{)}\PY{p}{)}
\end{Verbatim}

    \subsubsection{NOV = 12}\label{nov-12}

First we solve ESF of 12 variables with brute force, to confirm
correctness of the SGA solution.

    \begin{Verbatim}[commandchars=\\\{\}]
{\color{incolor}In [{\color{incolor}15}]:} \PY{n}{solve\PYZus{}and\PYZus{}print}\PY{p}{(}\PY{l+m+mi}{12}\PY{p}{)}
\end{Verbatim}

    \begin{Verbatim}[commandchars=\\\{\}]
nov12  =>  min fitness = 2     max fitness = 66
    \end{Verbatim}

    \paragraph{NOV = 12, MIN = 2, MAX = 66}\label{nov-12-min-2-max-66}

    \begin{Verbatim}[commandchars=\\\{\}]
{\color{incolor}In [{\color{incolor}16}]:} \PY{n}{opts} \PY{o}{=} \PY{p}{\PYZob{}} \PY{l+s}{\PYZsq{}}\PY{l+s}{num\PYZus{}eras}\PY{l+s}{\PYZsq{}}\PY{p}{:} \PY{l+m+mi}{25} \PY{p}{\PYZcb{}}
         \PY{n}{plot\PYZus{}esf\PYZus{}minmax}\PY{p}{(}\PY{l+m+mi}{12}\PY{p}{,} \PY{n}{min\PYZus{}ga\PYZus{}opts}\PY{o}{=}\PY{n}{opts}\PY{p}{,} \PY{n}{max\PYZus{}ga\PYZus{}opts}\PY{o}{=}\PY{n}{opts}\PY{p}{)}
\end{Verbatim}

    \begin{Verbatim}[commandchars=\\\{\}]
 minimization
 ============

Global optimum: 111001011011
Fitness: 2
Decoded: [1, 1, 1, -1, -1, 1, -1, 1, 1, -1, 1, 1]

 maximization
 ============

Global optimum: 111111111111
Fitness: 66
Decoded: [1, 1, 1, 1, 1, 1, 1, 1, 1, 1, 1, 1]
    \end{Verbatim}

    \begin{center}
    \adjustimage{max size={0.9\linewidth}{0.9\paperheight}}{SGA_files/SGA_34_1.png}
    \end{center}
    { \hspace*{\fill} \\}

    \subsubsection{NOV = 14}\label{nov-14}

    \begin{Verbatim}[commandchars=\\\{\}]
{\color{incolor}In [{\color{incolor}17}]:} \PY{n}{solve\PYZus{}and\PYZus{}print}\PY{p}{(}\PY{l+m+mi}{14}\PY{p}{)}
\end{Verbatim}

    \begin{Verbatim}[commandchars=\\\{\}]
nov14  =>  min fitness = 1     max fitness = 91
    \end{Verbatim}

    \paragraph{NOV = 14, MIN = 1, MAX = 91}\label{nov-14-min-1-max-91}

    \begin{Verbatim}[commandchars=\\\{\}]
{\color{incolor}In [{\color{incolor}18}]:} \PY{n}{plot\PYZus{}esf\PYZus{}minmax}\PY{p}{(}\PY{l+m+mi}{14}\PY{p}{,} \PY{n}{min\PYZus{}ga\PYZus{}opts}\PY{o}{=}\PY{p}{\PYZob{}}\PY{l+s}{\PYZsq{}}\PY{l+s}{num\PYZus{}eras}\PY{l+s}{\PYZsq{}}\PY{p}{:} \PY{l+m+mi}{25}\PY{p}{\PYZcb{}}\PY{p}{,} \PY{n}{max\PYZus{}ga\PYZus{}opts}\PY{o}{=}\PY{p}{\PYZob{}}\PY{l+s}{\PYZsq{}}\PY{l+s}{num\PYZus{}eras}\PY{l+s}{\PYZsq{}}\PY{p}{:} \PY{l+m+mi}{30}\PY{p}{\PYZcb{}}\PY{p}{)}
\end{Verbatim}

    \begin{Verbatim}[commandchars=\\\{\}]
 minimization
 ============

Global optimum: 11011011100011
Fitness: 1
Decoded: [1, 1, -1, 1, 1, -1, 1, 1, 1, -1, -1, -1, 1, 1]

 maximization
 ============

Global optimum: 11111111111111
Fitness: 91
Decoded: [1, 1, 1, 1, 1, 1, 1, 1, 1, 1, 1, 1, 1, 1]
    \end{Verbatim}

    \begin{center}
    \adjustimage{max size={0.9\linewidth}{0.9\paperheight}}{SGA_files/SGA_38_1.png}
    \end{center}
    { \hspace*{\fill} \\}

    \subsubsection{NOV = 27}\label{nov-27}

MIN and MAX not solved with brute force.

    \begin{Verbatim}[commandchars=\\\{\}]
{\color{incolor}In [{\color{incolor}19}]:} \PY{n}{opts} \PY{o}{=} \PY{p}{\PYZob{}} \PY{l+s}{\PYZsq{}}\PY{l+s}{num\PYZus{}eras}\PY{l+s}{\PYZsq{}}\PY{p}{:} \PY{l+m+mi}{50} \PY{p}{\PYZcb{}}
         \PY{n}{plot\PYZus{}esf\PYZus{}minmax}\PY{p}{(}\PY{l+m+mi}{27}\PY{p}{,} \PY{n}{min\PYZus{}ga\PYZus{}opts}\PY{o}{=}\PY{n}{opts}\PY{p}{,} \PY{n}{max\PYZus{}ga\PYZus{}opts}\PY{o}{=}\PY{n}{opts}\PY{p}{)}
\end{Verbatim}

    \begin{Verbatim}[commandchars=\\\{\}]
 minimization
 ============

Global optimum: 101100101011101111110010100
Fitness: 1
Decoded: [1, -1, 1, 1, -1, -1, 1, -1, 1, -1, 1, 1, 1, -1, 1, 1, 1, 1, 1, 1, -1,
          -1, 1, -1, 1, -1, -1]

 maximization
 ============

Global optimum: 000000000000000000000000000
Fitness: 351
Decoded: [-1, -1, -1, -1, -1, -1, -1, -1, -1, -1, -1, -1, -1, -1, -1, -1, -1, -1,
          -1, -1, -1, -1, -1, -1, -1, -1, -1]
    \end{Verbatim}

    \begin{center}
    \adjustimage{max size={0.9\linewidth}{0.9\paperheight}}{SGA_files/SGA_40_1.png}
    \end{center}
    { \hspace*{\fill} \\}

    \subsubsection{NOV = 35}\label{nov-35}

    \begin{Verbatim}[commandchars=\\\{\}]
{\color{incolor}In [{\color{incolor}20}]:} \PY{n}{plot\PYZus{}esf\PYZus{}minmax}\PY{p}{(}\PY{l+m+mi}{35}\PY{p}{)}
\end{Verbatim}

    \begin{Verbatim}[commandchars=\\\{\}]
 minimization
 ============

Global optimum: 10100100000101011011010011101111111
Fitness: 5
Decoded: [1, -1, 1, -1, -1, 1, -1, -1, -1, -1, -1, 1, -1, 1, -1, 1, 1, -1, 1, 1,
          -1, 1, -1, -1, 1, 1, 1, -1, 1, 1, 1, 1, 1, 1, 1]

 maximization
 ============

Global optimum: 00000000000000000000000000000000000
Fitness: 595
Decoded: [-1, -1, -1, -1, -1, -1, -1, -1, -1, -1, -1, -1, -1, -1, -1, -1, -1, -1, -1,
          -1, -1, -1, -1, -1, -1, -1, -1, -1, -1, -1, -1, -1, -1, -1, -1]
    \end{Verbatim}

    \begin{center}
    \adjustimage{max size={0.9\linewidth}{0.9\paperheight}}{SGA_files/SGA_42_1.png}
    \end{center}
    { \hspace*{\fill} \\}

    \subsubsection{NOV = 100}\label{nov-100}

Maximization requires more than 100 eras for a good probability of
converging to 4950, which appears to the be maximum value.

    \begin{Verbatim}[commandchars=\\\{\}]
{\color{incolor}In [{\color{incolor}21}]:} \PY{n}{plot\PYZus{}esf\PYZus{}minmax}\PY{p}{(}\PY{l+m+mi}{100}\PY{p}{,} \PY{n}{max\PYZus{}ga\PYZus{}opts}\PY{o}{=}\PY{p}{\PYZob{}} \PY{l+s}{\PYZsq{}}\PY{l+s}{num\PYZus{}eras}\PY{l+s}{\PYZsq{}}\PY{p}{:} \PY{l+m+mi}{250} \PY{p}{\PYZcb{}}\PY{p}{)}
\end{Verbatim}

    \begin{Verbatim}[commandchars=\\\{\}]
 minimization
 ============

Global optimum: 000000011111110010100100001001101101000110010110000101001001110101
                1010001101011111000101010001001100
Fitness: 0
Decoded: [-1, -1, -1, -1, -1, -1, -1, 1, 1, 1, 1, 1, 1, 1, -1, -1, 1, -1, 1, -1, -1, 1,
          -1, -1, -1, -1, 1, -1, -1, 1, 1, -1, 1, 1, -1, 1, -1, -1, -1, 1, 1, -1, -1,
          1, -1, 1, 1, -1, -1, -1, -1, 1, -1, 1, -1, -1, 1, -1, -1, 1, 1, 1, -1, 1, -1,
          1, 1, -1, 1, -1, -1, -1, 1, 1, -1, 1, -1, 1, 1, 1, 1, 1, -1, -1, -1, 1, -1, 1,
          -1, 1, -1, -1, -1, 1, -1, -1, 1, 1, -1, -1]

 maximization
 ============

Global optimum: 000000000000000000000000000000000000000000000000000000000000000000000
                0000000000000000000000000000000
Fitness: 4950
Decoded: [-1, -1, -1, -1, -1, -1, -1, -1, -1, -1, -1, -1, -1, -1, -1, -1, -1, -1, -1,
          -1, -1, -1, -1, -1, -1, -1, -1, -1, -1, -1, -1, -1, -1, -1, -1, -1, -1, -1,
          -1, -1, -1, -1, -1, -1, -1, -1, -1, -1, -1, -1, -1, -1, -1, -1, -1, -1, -1,
          -1, -1, -1, -1, -1, -1, -1, -1, -1, -1, -1, -1, -1, -1, -1, -1, -1, -1, -1,
          -1, -1, -1, -1, -1, -1, -1, -1, -1, -1, -1, -1, -1, -1, -1, -1, -1, -1, -1,
          -1, -1, -1, -1, -1]
    \end{Verbatim}

    \begin{center}
    \adjustimage{max size={0.9\linewidth}{0.9\paperheight}}{SGA_files/SGA_44_1.png}
    \end{center}
    { \hspace*{\fill} \\}

    \subsubsection{NOV = 200}\label{nov-200}

All of the above benchmarks take no time at all, 5 seconds in the worst
case.

From the lower variable numbers, it can be observed that increasing the
variable size makes maximization problems more difficult. Letting the GA
run for longer will usually help it, but the number of eras is the
dominant variable for time required for the SGA to run. For a lot of
variables, we need to let SGA run for many eras, with a large
population, to be confident that we've found the maximum.

    \paragraph{NOV = 200, minimization}\label{nov-200-minimization}

    \begin{Verbatim}[commandchars=\\\{\}]
{\color{incolor}In [{\color{incolor}22}]:} \PY{k+kn}{import} \PY{n+nn}{time}

         \PY{n}{ga\PYZus{}opts} \PY{o}{=} \PY{p}{\PYZob{}}
             \PY{l+s}{\PYZsq{}}\PY{l+s}{chromosome\PYZus{}length}\PY{l+s}{\PYZsq{}}\PY{p}{:} \PY{l+m+mi}{200}\PY{p}{,} \PY{c}{\PYZsh{} nov = 200,}
             \PY{l+s}{\PYZsq{}}\PY{l+s}{num\PYZus{}eras}\PY{l+s}{\PYZsq{}}\PY{p}{:} \PY{l+m+mi}{50}\PY{p}{,}
         \PY{p}{\PYZcb{}}
         \PY{n}{start} \PY{o}{=} \PY{n}{time}\PY{o}{.}\PY{n}{time}\PY{p}{(}\PY{p}{)}
         \PY{n}{plot\PYZus{}ga}\PY{p}{(}\PY{n}{esf}\PY{p}{,} \PY{n}{esf\PYZus{}decoder}\PY{p}{,} \PY{n}{min\PYZus{}or\PYZus{}max}\PY{o}{=}\PY{n}{MIN}\PY{p}{,} \PY{n}{ga\PYZus{}opts}\PY{o}{=}\PY{n}{ga\PYZus{}opts}\PY{p}{,} \PY{n}{title}\PY{o}{=}\PY{l+s}{\PYZdq{}}\PY{l+s}{ESF (NOV=200) (minimization)}\PY{l+s}{\PYZdq{}}\PY{p}{)}

         \PY{n+nb}{print}\PY{p}{(}\PY{l+s}{\PYZdq{}}\PY{l+s+se}{\PYZbs{}n}\PY{l+s}{ESF (NOV=200) minimization took \PYZob{}:.3f\PYZcb{} s}\PY{l+s}{\PYZdq{}}\PY{o}{.}\PY{n}{format}\PY{p}{(}\PY{n}{time}\PY{o}{.}\PY{n}{time}\PY{p}{(}\PY{p}{)} \PY{o}{\PYZhy{}} \PY{n}{start}\PY{p}{)}\PY{p}{)}
\end{Verbatim}

    \begin{Verbatim}[commandchars=\\\{\}]
Global optimum: 11101111101000110100001111111001101001111011010101011100111110101000100
                00110001011111110100010101000101000000110011111001111110011011111111000
                1101000011001010101110111000000010010011010101100001000111
Fitness: 2
Decoded: [1, 1, 1, -1, 1, 1, 1, 1, 1, -1, 1, -1, -1, -1, 1, 1, -1, 1, -1, -1, -1, -1,
          1, 1, 1, 1, 1, 1, 1, -1, -1, 1, 1, -1, 1, -1, -1, 1, 1, 1, 1, -1, 1, 1, -1,
          1, -1, 1, -1, 1, -1, 1, 1, 1, -1, -1, 1, 1, 1, 1, 1, -1, 1, -1, 1, -1, -1, -1,
          1, -1, -1, -1, -1, 1, 1, -1, -1, -1, 1, -1, 1, 1, 1, 1, 1, 1, 1, -1, 1, -1,
          -1, -1, 1, -1, 1, -1, 1, -1, -1, -1, 1, -1, 1, -1, -1, -1, -1, -1, -1, 1, 1,
          -1, -1, 1, 1, 1, 1, 1, -1, -1, 1, 1, 1, 1, 1, 1, -1, -1, 1, 1, -1, 1, 1, 1,
          1, 1, 1, 1, 1, -1, -1, -1, 1, 1, -1, 1, -1, -1, -1, -1, 1, 1, -1, -1, 1, -1,
          1, -1, 1, -1, 1, 1, 1, -1, 1, 1, 1, -1, -1, -1, -1, -1, -1, -1, 1, -1, -1, 1,
          -1, -1, 1, 1, -1, 1, -1, 1, -1, 1, 1, -1, -1, -1, -1, 1, -1, -1, -1, 1, 1, 1]

ESF (NOV=200) minimization took 10.856 s
    \end{Verbatim}

    \begin{center}
    \adjustimage{max size={0.9\linewidth}{0.9\paperheight}}{SGA_files/SGA_47_1.png}
    \end{center}
    { \hspace*{\fill} \\}

    \paragraph{NOV = 200, maximization}\label{nov-200-maximization}

Computing the maximum of ESF of 20 variables (wherein we assume that
each term of the ESF is 1), we obtain 19900.

After running for almost 13 minutes, my SGA implementation didn't find
the maximum, but it came sort of close. In fact, since it is evident
that 000\ldots{}0 or 111\ldots{}1 will yield the optimum, we can
determine that the solution found by SGA was off by 5 bits, or 2.5\% of
the variables.

(Full disclosure: I ran this at least 3 times with the parameters below.
One run of it came as close as 3 bits away from optimality.)

    \begin{Verbatim}[commandchars=\\\{\}]
{\color{incolor}In [{\color{incolor}24}]:} \PY{n}{ga\PYZus{}opts} \PY{o}{=} \PY{p}{\PYZob{}}
             \PY{l+s}{\PYZsq{}}\PY{l+s}{chromosome\PYZus{}length}\PY{l+s}{\PYZsq{}}\PY{p}{:} \PY{l+m+mi}{200}\PY{p}{,} \PY{c}{\PYZsh{} nov = 200,}
             \PY{l+s}{\PYZsq{}}\PY{l+s}{crossover\PYZus{}probability}\PY{l+s}{\PYZsq{}}\PY{p}{:} \PY{l+m+mf}{0.4}\PY{p}{,}
             \PY{l+s}{\PYZsq{}}\PY{l+s}{population\PYZus{}size}\PY{l+s}{\PYZsq{}}\PY{p}{:} \PY{l+m+mi}{60}\PY{p}{,}
             \PY{l+s}{\PYZsq{}}\PY{l+s}{num\PYZus{}eras}\PY{l+s}{\PYZsq{}}\PY{p}{:} \PY{l+m+mi}{1000}\PY{p}{,}
         \PY{p}{\PYZcb{}}

         \PY{n}{start} \PY{o}{=} \PY{n}{time}\PY{o}{.}\PY{n}{time}\PY{p}{(}\PY{p}{)}
         \PY{n}{plot\PYZus{}ga}\PY{p}{(}\PY{n}{esf}\PY{p}{,} \PY{n}{esf\PYZus{}decoder}\PY{p}{,} \PY{n}{min\PYZus{}or\PYZus{}max}\PY{o}{=}\PY{n}{MAX}\PY{p}{,} \PY{n}{ga\PYZus{}opts}\PY{o}{=}\PY{n}{ga\PYZus{}opts}\PY{p}{,}
                 \PY{n}{title}\PY{o}{=}\PY{l+s}{\PYZdq{}}\PY{l+s}{ESF (NOV=200) (maximization)}\PY{l+s}{\PYZdq{}}\PY{p}{,} \PY{n}{legend}\PY{o}{=}\PY{k}{False}\PY{p}{)}
         \PY{n}{total} \PY{o}{=} \PY{n}{time}\PY{o}{.}\PY{n}{time}\PY{p}{(}\PY{p}{)} \PY{o}{\PYZhy{}} \PY{n}{start}

         \PY{n+nb}{print}\PY{p}{(}\PY{l+s}{\PYZdq{}}\PY{l+s+se}{\PYZbs{}n}\PY{l+s}{ESF (NOV=200) maximization took \PYZob{}:.3f\PYZcb{} s (\PYZob{}:.2f\PYZcb{} m)}\PY{l+s}{\PYZdq{}}
               \PY{o}{.}\PY{n}{format}\PY{p}{(}\PY{n}{total}\PY{p}{,} \PY{n}{total}\PY{o}{/}\PY{l+m+mi}{60}\PY{p}{)}\PY{p}{)}
\end{Verbatim}

    \begin{Verbatim}[commandchars=\\\{\}]
Global optimum: 00000000000000000000000000000000100001000000000000000000000000001000100
                00000000000000000000000000000000000000000000000000000000000000000000000
                0000001000000000000000000000000000000000000000000000000000
Fitness: 17950
Decoded: [-1, -1, -1, -1, -1, -1, -1, -1, -1, -1, -1, -1, -1, -1, -1, -1, -1, -1, -1, -1,
          -1, -1, -1, -1, -1, -1, -1, -1, -1, -1, -1, -1, 1, -1, -1, -1, -1, 1, -1, -1,
          -1, -1, -1, -1, -1, -1, -1, -1, -1, -1, -1, -1, -1, -1, -1, -1, -1, -1, -1, -1,
          -1, -1, -1, -1, 1, -1, -1, -1, 1, -1, -1, -1, -1, -1, -1, -1, -1, -1, -1, -1,
          -1, -1, -1, -1, -1, -1, -1, -1, -1, -1, -1, -1, -1, -1, -1, -1, -1, -1, -1, -1,
          -1, -1, -1, -1, -1, -1, -1, -1, -1, -1, -1, -1, -1, -1, -1, -1, -1, -1, -1, -1,
          -1, -1, -1, -1, -1, -1, -1, -1, -1, -1, -1, -1, -1, -1, -1, -1, -1, -1, -1, -1,
          -1, -1, -1, -1, -1, -1, -1, -1, 1, -1, -1, -1, -1, -1, -1, -1, -1, -1, -1, -1,
          -1, -1, -1, -1, -1, -1, -1, -1, -1, -1, -1, -1, -1, -1, -1, -1, -1, -1, -1, -1,
          -1, -1, -1, -1, -1, -1, -1, -1, -1, -1, -1, -1, -1, -1, -1, -1, -1, -1, -1, -1]

ESF (NOV=200) maximization took 775.941 s (12.93 m)
    \end{Verbatim}

    \begin{center}
    \adjustimage{max size={0.9\linewidth}{0.9\paperheight}}{SGA_files/SGA_49_1.png}
    \end{center}
    { \hspace*{\fill} \\}


    % Add a bibliography block to the postdoc



    \end{document}
